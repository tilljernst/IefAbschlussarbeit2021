% Ausgangslage
% ************
\section{Ausgangslage} \label{Ausgangslage}
\subsection{Rahmenbedingungen}
Die Behandlung des Fall SK fand und findet am \ac{kjpd} des Kantons Schaffhausen im Rahmen einer ambulanten psychiatrischen Behandlung statt. Der \ac{kjpd} Schaffhausen umfasst ein Team von insgesamt 12 Psycholog*innen und 3 Ärzte und ist an den Spitälern Schaffhausen angegliedert. Der \ac{kjpd} besitzt den Auftrag ein niederschwelliges, entwicklungs- und bedarfsgerechtes Angebot zur Abklärung und Behandlung für Kinder und Jugendliche mit psychosozial bedingten Entwicklungsgefährdungen und psychischen Störungen im Kanton Schaffhausen bereitzuhalten. Ich bin als Psychologe in therapeutischer Weiterbildung zu einem Pensum von 70\% im Bereich Abklärung und Therapie Kinder- und Jugendlichen angestellt. In dieser Funktion bin ich einerseits dem leitenden Arzt, sowie der leitenden Psychologin direkt unterstellt. Mit dem leitende Arzt habe ich wöchentliche Supervisionssitzungen für die Fallbesprechung und das Besprechen von Berichten, wie Auftragsklärung,  Abklärungsbefunde, IV Arztberichte, Krankenkassenschreiben, etc. Mit der leitende Psychologin habe ich zu tun, wenn es um meine Ausbildung und die administrativen Belange wie Gehalt, Weiterbildungskosten, Kompensation und Ferien geht.  

Aufgrund der Grösse des \ac{kjpd}s und des Versorgungsauftrags des Kantons gestaltet sich der Aufgabenbereich für mich als einzelner Behandler sehr breit. Es werden jedoch gewisse Spezialgebiete angeboten, die mit Hilfe sogenannter Kompetenzbereiche abgedeckt werden. Ich bin aufgrund der aktuell laufenden Weiterbildung am \ac{ief} keinem expliziten Kompetenzbereich zugeordnet, übernehme jedoch je nach Situation Fälle, die im Schnittbereich dieser Bereich angesiedelt sind. Die nötige Abstützung und Sicherheit kann ich mir aufgrund der internen Struktur der kurzen Meldewege holen, indem ich im engmaschigen Austausch stehe, sowie Intervisiongespräche und Coaching mit den Fachpersonen aus den entsprechenden Bereichen nach Bedarf führen kann. Diese Struktur vermittelt mir ein Gefühlt der Sicherheit durch den guten Rückenhalt, wodurch ich mir auch Fälle zutraue, die für mich neu und komplex sind.

Die Zusammenarbeit mit anderen Berufsgruppen innerhalb des \ac{kjpd}s findet vorwiegend zwischen psychologisch und medizinisch geschultem Personal statt. Aufgrund der institutionellen Stellung einer ambulanten psychiatrischen Behandlung kommt es naturbedingt zu Diskrepanzen zwischen dieser und meiner sich weiterentwickelnden systemischen Sichtweise. Es fällt mir nicht immer einfach, die systemische Sicht als Psychotherapeut und die Sichtweise der Institution, resp. des Gesundheitswesens im Allgemeinen, auf einen Nenner zu bringen. Aufgrund der Möglichkeit, sehr flexibel mit den einzelnen Fällen vorzugehen und die eigenen Vorlieben für die Gestaltung der Behandlung einfliessen zu lassen, befinden sich diese Diskrepanz jedoch auf einem minimalen Niveau. Ein bleibende  Unsicherheit meinerseits ergibt sich aufgrund der ursprünglichen psychiatrischen Sichtweise einer defizitorientierten Einteilung, resp. die diagnostische Einschätzung mittels  ICD-10, die nach wie vor einen hohen Stellenwert in der Behandlung am \ac{kjpd} einnimmt. Unsicherheit insofern, als dass ich mich aktuell in meiner Weiterbildung im Modul Grundlagen befinde und die Störungsspezifischen Vorlesungen vor mir liegen. Aktuell versuche ich die defizitorientierte Diagnostik als notwendige Vorgehensweise für die Kostenübernahme durch die Krankenkassen oder anderweitige Kostenträger anzusehen. Vielmehr versuche ich mir die Frage zu stellen, wem diese Diagnose etwas nützen soll. In Absprache mit meinem direkten Vorgesetzten, dem leitenden Arzt, kann dies im Extremfall soweit führen, dass wir uns einer mögliche Diagnose nach ICD-10 enthalten, wenn wir der Meinung sind, dass diese dem System mehr schaden zufügt. Idealerweise kann ich diese Entscheidung mit den Klienten zusammen besprechen, um gemeinsam eine Lösung zu finden. Dieses Vorgehen schätze ich sehr.

Die Zusammenarbeit mit anderen Berufsgruppen ausserhalb des \ac{kjpd}s, insbesondere derjenigen aus dem Bereich der Schule, gestaltet sich aufgrund des oben angesprochenen Behandlungs- und Störungsverständnis je nach Fall unterschiedlich. Insbesondere mit der Verwandten Institution \ac{sab} wird die Zusammenarbeit dadurch erschwert, da diese den Fokus stärker auf eine Störung im ursprünglichen Sinn, nämlich gekoppelt an eine Diagnose verstehen. Aus eigener Erfahrung kommt es dabei auf mein Geschick in der Gesprächsführung an, ob eine möglichst ressourcenorientierte Vorgehensweise eingeschlagen werden kann. Als hilfreich hat sich die möglichst wertneutrale Auseinandersetzung herausgestellt. Zudem scheint ein möglichst verhaltensbeschreibendes Vorgehen auf Akzeptanz zu stossen, wenn der Hintergedanke einer solchen Sichtweise erläutert wird. Nicht die Störung an und für sich soll aus meiner Sicht im Fokus stehen, sondern das gezeigte Verhalten, um gemeinsam mit den involvierten Fachpersonen eine passende Lösung zu finden, angelehnt an das lösungsfokussierte Vorgehens nach \citeA{Shazer1986}. 
 
\subsection{Intakte} 
Die Anmeldung von SK am KJPD erfolgte anfangs 2019 auf Initiative der Mutter. SK war zum damaligen Zeitpunkt 16 Jahre alt. Damals arbeitete ich seit etwas mehr als einem halben Jahr am KJPD und hatte mit meiner Weiterbildung am \ac{ief} zum systemischen Psychotherapeuten noch nicht begonnen. 

Aufgrund der Anmeldung von SK war mir bekannt, dass es sich um eine erneute Anmeldung am KJPD handelte. Der Fall wurde mir damals direkt vom Chefarzt zugewiesen, da dies zu jener Zeit hier so üblich war. Gemäss den Vorabinformationen war mit bekannt, dass es sich bei SK um den Wunsch einer ambulanten Therapieübernahme nach einem Klinikaufenthalt handelte. Die Informationen aus der ersten Behandlung habe ich absichtlich nicht eingesehen, um mir ein möglichst neutrales Bild des Anliegens zu verschaffen. Ich hatte die Befürchtung, dass ich mit weiteren Informationen aus der \ac{kg} beeinflusst werden könnte und somit voreingenommen in das erste Gespräch einsteigen würde. Ohne mir dessen bewusst zu sein, versuchte ich den Ansatz des \textbf{Not Knowing} zu praktizieren \cite{Anderson1992}. Also nicht ich als Therapeut weiss was bei SK los ist und was gut für sie ist, sondern SK selber. Dieser Ansatz fiel mir zu jener Zeit relativ leicht, da ich als Therapieneuling tatsächlich noch keine grosse Erfahrung hatte und noch wenig vertraut war von der psychiatrischen Behandlungspraxis am KJPD, somit wusste ich tatsächlich sehr wenig. 

In der Rückschau ergeben sich jedoch Hinweise auf Hypothesen, die ich trotz weniger Informationen aufgrund der Anmeldung bereits formulierte. Einerseits konnte ich auf der Anmeldung sehen, dass der Klinikaufenthalt bei SK wohl aufgrund von Ängsten stattgefunden hatte. Zudem ging ich davon aus, dass die Eltern getrennt leben, da der Vater über eine separate Anmeldung informiert wurde. Weiter schloss ich aus dem Anmeldegrund, dass SK mit der damaligen ambulanten Therapeutin nicht zufrieden war, da sie sich für eine ambulanter Behandlung am KJPD meldete. 

Mein erster Eindruck nach dem Eingang der Anmeldung war, dass ich bei SK im weitesten Sinne einen klientzentrierten Ansatzes nach \citeA{Rogers1993} versuchen wollte. Dies wohl aus dem einfachen Grund, da ich mich vor meiner Weiterbildung am \ac{ief} in meinem Studium an der ZHAW vertieft mit diesem Ansatz auseinander gesetzt habe und ich mir schlicht nichts anderes vorstellen konnte. Ohne Erfahrung fühlte ich mich mit diesem Ansatz sicher genug, eine mögliche Therapie zu übernehmen.

\subsection{Bestehende explizite und implizite Aufträge} 
Im Vorfeld konnte ich folgende expliziten und impliziten Aufträge ausmachen: Aus Sicht der Institution besteht ein expliziter Auftrag im Sinne der Sicherstellung der Psychiatrischen Versorgung im Kanton. Eine Anmeldung am \ac{kjpd} muss zwingend geprüft werden. Damals war es hier am Haus usus, das Anliegen nach der Anmeldung mittels \ac{akg} zu prüfen (aktuell wird vor der eigentlichen Anmeldung immer ein telefonisches Vorgespräch durch ein psychologisch geschultes Personal geführt, um diese Prüfung vorzunehmen und gegebenenfalls an eine andere Stelle weiterzuleiten). Ein weiterer expliziter Auftrag konnte ich dem Anmeldeblatt entnehmen, auf dem der Wunsch nach einer Therapieübernahme im ambulanten Setting genannt wurde. Ansonsten war mir im Vorfeld kein weiterer Auftrag bekannt. Anhand der Einteilung nach \citeA{Schwing2014}, welcher nach Anlass, Anliegen, Auftrag und Kontrakt unterscheidet, konnte ich zu den Einzelnen Punkte jedoch bereits implizite Aufträge erahnen: Als Anlass konnte ich aus der Anmeldung annehmen, dass SK einen Aufenthalt in einer Klinik hinter sich hatte und deshalb Unterstützung für die Nachbetreuung suchte, welche sie bei der ambulanten Behandlerin nicht zufriedenstellend erhielt. Beim Anliegen und den Erwartungen konnte ich implizit den Wunsch nach einer sich verändernden Nachbetreuung ausmachen, da offensichtlich diese nicht zu den erhofften Veränderungen geführt hatte. Beim Punkt Auftrag stand das explizite Anliegen einer Therapieübernahme. Was jedoch genau von mir erwartet wurde und was ich in dem Fall SK zu tun hatte, war vor der Auftragsklärung nicht bekannt. Im Vorfeld hatten wir somit bereits etwas ähnliches wie ein Kontrakt, denn Grundsätzlich können wir das Anliegen einer Therapieübernahme am \ac{kjpd} anbieten. Angebot und Anliegen stimmten somit anhand der Vorinformationen grundsätzlich überein.  

Wie ich weiter unten im Kapitel \titleref{Auftragsklärung} schildern werde, musste ich diese Aufträge (explizite und implizite) teilweise revidieren, da sich anhand der von SK, insbesondere der Mutter, geäusserten Zieldefinitionen leicht angepasste Aufträge ergeben haben. An diesem Punkt sei schon verraten, dass sich die Aufträge im Verlauf der Behandlung noch einige Male veränderten. Zudem konnte ich vor dem \ac{akg} noch keinen expliziten Auftrag von SK ausfindig machen, da die Anmeldung durch die Mutter vorgenommen wurde. 

\subsection{Settingwahl fürs Erstgespräch} 
Für das Erstgespräch, oder eben das \ac{akg}, entschied ich mich die Mutter und die Klientin SK ohne den Vater einzuladen. Das ist etwas speziell und entspricht nicht den Gepflogenheiten am \ac{kjpd}. Bei minderjährigen Klienten, was SK zum damaligen Zeitpunkt noch war, müssten eigentlich beide Elternteile zum Erstgespräch erscheinen. Dies rein aus rechtlicher Sicht, da beiden Eltern ihr Einverständnis für eine Behandlung geben müssen. Ausser wenn ein Elternteil das alleinige Sorgerecht inne hat. Das war bei SK jedoch nicht der Fall. Aus heutiger Sicht würde ich dies nicht mehr so machen und auf die Anwesenheit des Vaters bestehen. Damals war ich jedoch noch neu und  empfand es als naheliegend, nur die Mutter und die Tochter aufgrund der Anmeldung einzuladen. Mir war zudem nicht bewusst, wie wichtig beide Elternteile für eine saubere Auftragsklärung sind und mir durch mein Versäumnis wichtige Informationen entgehen würden.










