% Ausgangslage
% ************
\section{Ausgangslage - 3 Seiten}
\subsection{Rahmenbedingungen}
Die Behandlung des Fall SK fand und findet am \ac{kjpd} des Kantons Schaffhausen im Rahmen einer ambulanten psychiatrischen Behandlung statt. Der \ac{kjpd} Schaffhausen umfasst ein Team von insgesamt 12 Psycholog*innen und 3 Ärzte und ist an den Spitälern Schaffhausen angegliedert. Der \ac{kjpd} besitzt den Auftrag ein niederschwelliges, entwicklungs- und bedarfsgerechtes Angebot zur Abklärung und Behandlung für Kinder und Jugendliche mit psychosozial bedingten Entwicklungsgefährdungen und psychischen Störungen im Kanton Schaffhausen bereitzuhalten. Ich bin als Psychologe in therapeutischer Weiterbildung zu einem Pensum von 70\% im Bereich Abklärung und Therapie Kinder- und Jugendlichen angestellt. In dieser Funktion bin ich einerseits dem leitenden Arzt, sowie der leitenden Psychologin direkt unterstellt. Mit dem leitende Arzt habe ich wöchentliche Supervisionssitzungen für die Fallbesprechung und das Besprechen von Berichten, wie Auftragsklärungsg,  Abklärungsbefunde, IV Arztberichte, Krankenkassenschreiben, etc. Mit der leitende Psychologin habe ich zu tun, wenn es um meine Ausbildung und die administrativen Belange geht, wie Gehalt, Weiterbildungskosten, Kompensation und Ferien. 

Das Leistungsangebot am \ac{kjpd} richtet sich an Kinder und Jugendliche und deren Eltern, die wohnhaft im Kanton Schaffhausen sind. Das Vorgehen bei einer Neuanmeldung umfasst grundsätzlich ein Erstgespräch, in dem die Anliegen und Schwierigkeiten besprochen werden, um eine gemeinsame Sichtweise zwischen Klient*in und Behandler*in zu entwickeln. Dieser Ablauf ist jedoch mehr als eine Art Richtlinie zu verstehen. Als Fallführer ist er mir überlassen, wie ich mein Vorgehen gestalte. Für mich als Psychologe in Weiterbildung ergibt sich aus diesem Vorgehen eine gewisse Sicherheit, die mir als Leitplanke dient und an der ich mich vortasten kann. 

Aufgrund der Grösse des \ac{kjpd}s und des Versorgungsauftrags des Kantons gestaltet sich der Aufgabenbereich für mich als einzelner Behandler sehr breit. Es werden jedoch gewisse Spezialgebiete angeboten, die mit Hilfe sogenannter Kompetenzbereiche abgedeckt werden. Ich bin aufgrund der aktuell laufenden Weiterbildung am IEF keinem expliziten Kompetenzbereich zugeordnet, übernehme jedoch je nach Situation Fälle, die im Schnittbereich dieser Bereich angesiedelt sind. Die nötige Abstützung und Sicherheit kann ich mir aufgrund der internen Struktur der kurzen Meldewege holen, indem ich im engmaschigen Austausch stehe, sowie Intervisiongespräche und Coaching mit den Fachpersonen aus den entsprechenden Bereichen nach Bedarf führen kann. Diese Struktur vermittelt mir ein Gefühlt der Sicherheit durch den guten Rückenhalt, wodurch ich mir auch Fälle zutraue, die für mich neu und komplex sind. 

Die Zusammenarbeit mit anderen Berufsgruppen innerhalb des \ac{kjpd}s findet vorwiegend zwischen psychologisch und medizinisch geschultem Personal statt. Aufgrund der institutionellen Stellung einer ambulanten psychiatrischen Behandlung kommt es naturbedingt zu Diskrepanzen zwischen dieser und meiner sich weiterentwickelnden systemischen Sichtweise aufgrund der Weiterbildung. Es fällt mir nicht immer einfach, die systemische Sicht als Psychotherapeut und die Sichtweise der Institution, resp. des Gesundheitswesens im Allgemeinen, auf einen Nenner zu bringen. Aufgrund der Möglichkeit, sehr flexibel mit den einzelnen Fällen vorzugehen und die eigenen Vorlieben für die Gestaltung der Behandlung einfliessen zu lassen, befinden sich diese Diskrepanz jedoch auf einem minimalen Niveau. Ein bleibende  Unsicherheit meinerseits ergibt sich aufgrund der ursprünglichen psychiatrischen Sichtweise einer defizitorientierten Einteilung, resp. die diagnostische Einschätzung mittels  ICD-10, die nach wie vor einen hohen Stellenwert in der Behandlung am \ac{kjpd} einnimmt. Unsicherheit insofern, als dass ich mich aktuell in meiner Weiterbildung im Modul Grundlagen befinde und die Störungsspezifischen Vorlesungen noch vor mir liegen. Aktuell versuche ich die defizitorientierte Diagnostik als notwendige Vorgehensweise für die Kostenübernahme durch die Krankenkassen oder anderweitige Kostenträger anzusehen und weniger als Behandlungsgrundlage für das weitere Vorgehen zu verstehen. Ich versuche mir die Frage zu stellen, wem diese Diagnose etwas nützen soll. In Absprache mit meinem direkten Vorgesetzten kann dies im Extremfall soweit führen, dass wir uns einer mögliche Diagnose nach ICD-10 enthalten, wenn wir der Meinung sind, dass diese dem System mehr schaden zufügt. Idealerweise kann ich diese Entscheidung mit den Klienten zusammen besprechen, um gemeinsam eine Lösung zu finden. 

Die Zusammenarbeit mit anderen Berufsgruppen ausserhalb des \ac{kjpd}s, insbesondere derjenigen aus dem Bereich der Schule, gestaltet sich aufgrund des oben angesprochenen Behandlungs- und Störungsverständnis je nach Fall unterschiedlich. Insbesondere mit der Verwandten \ac{sab} wird die Zusammenarbeit dadurch erschwert, da diese den Fokus stärker auf eine Störung im ursprünglichen Sinn, nämlich gekoppelt an eine Diagnose, verstehen, um darauf ihre Arbeit und ihr Vorgehen zu stützen. Aus eigener Erfahrung kommt es dabei auf mein Geschick in der Gesprächsführung an, ob eine möglichst ressourcenorientierte Vorgehensweise eingeschlagen werden kann. Als hilfreich hat sich die möglichst wertneutrale Auseinandersetzung herausgestellt, ohne die eigene Sichtweise dem Gegenüber aufzwängen zu wollen. Zudem scheint ein möglichst verhaltensbeschreibendes Vorgehen auf Akzeptanz zu stossen, wenn der Hintergedanke einer solchen Sichtweise erläutert wird. Nicht die Störung an und für sich soll im Fokus stehen, sondern das Verhalten, um möglichst gemeinsam mit den involvierten Fachpersonen eine passende Lösung zu finden. Ganz im Sinne von einem lösungsfokussierten Vorgehens \cite{Shazer1986}. 
 
\subsection{Intakte} 
Die Anmeldung von SK am KJPD erfolgte anfangs 2019 auf Initiative der Mutter. SK war zum damaligen Zeitpunkt 16 Jahre alt. Damals arbeitete ich unmittelbar nach meinem Abschluss zum Psychologen M.Sc. seit etwas mehr als einem halben Jahr am KJPD und hatte mit meiner Weiterbildung am IEF zum systemischen Psychotherapeuten noch nicht begonnen. 

Aufgrund der Anmeldung von SK war bekannt, dass es sich um eine erneute Anmeldung am KJPD handelte. Der Fall wurde mir damals direkt vom Chefarzt zugewiesen, da dies zu jener Zeit so üblich war (alle Neuanmeldungen wurden jener Zeit vom Chefarzt gesichtet und im Anschluss an die Fallführer verteilt). Gemäss den Vorabinformationen aus der Anmeldung ging hervor, dass es sich bei SK um den Wunsch einer ambulanten Therapieübernahme handelte. Zudem wurde vermerkt, dass SK nach einem Klinikaufenthalt von einer Therapeutin ambulant behandelt wurde und den Wunsch äusserte, sich am KJPD für eine weiterführende ambulante Therapie zu melden.

Zudem wären mir weitere Informationen aus der bereits bestehenden \ac{kg} vorgelegen. Diese habe ich jedoch absichtlich nicht eingesehen, um mir ein möglichst voreingenommenes Bild des Anliegens zu verschaffen. Zur Zeit der Fallübernahme praktizierte ich dieses Vorgehen, da ich den Klienten möglichst unvoreingenommen gegenübertreten wollte und möglichst offen bleiben wollte. Ich hatte die Befürchtung mit weiteren Informationen aus der \ac{kg} beeinflusst zu werden und weniger neutral in das erste Gespräch einsteigen zu können. Ohne es aktiv und bewusst schon zu wissen, versuchte ich den Ansatz des \textbf{Not Knowing} zu praktizieren \cite{Anderson1992}. Also nicht ich als Therapeut weiss was bei SK los ist und was gut für sie ist, sondern SK selber. Dieser Ansatz viel mir zu jener Zeit relativ leicht, da ich als Therapieneuling tatsächlich noch keine grosse Erfahrung hatte und noch wenig vertraut war von der psychiatrischen Behandlungspraxis am KJPD. 

In der Rückschau ergeben sich jedoch Hinweise auf Hypothesen, die ich trotz weniger Informationen aufgrund der Anmeldung bereits formulierte. Einerseits konnte ich sehen, dass der Klinikaufenthalt bei SK wohl aufgrund von Ängsten stattgefunden hatte, da dies auf dem Anmeldeblatt so vermerkt war. Zudem ging ich davon aus, dass die Eltern getrennt leben, da gemäss Notiz auf der Anmeldung geschrieben stand, dass der Vater über die Anmeldung informiert wurde und ich automatisch von getrennt lebenden Elternteilen ausging. Weiter schloss ich aus dem Anmeldegrund, dass SK mit der aktuellen Therapeutin nicht zufrieden ist, da sie erneut nach ambulanter Behandlung am KJPD gelangen wollte. 

Mein erster Eindruck war, nachdem ich die Anmeldung in meinem Fach hatte, dass ich bei SK über eine Gesprächstherapie im weitesten Sinne eines klientzentrierten Ansatzes nach \citeA{Rogers1993} wohl gut bedient wäre. Dies wohl aus dem einfachen Grund, weil ich mich vor meiner Weiterbildung am IEF in meinem Studium an der ZHAW vertieft mit diesem Ansatz auseinander gesetzt habe und ich mir schlicht nichts anderes vorstellen konnte, wo ich mich sicher genug fühlte eine mögliche Therapie zu übernehmen. Ich ging  davon aus, dass eine Übernahme möglich und auch von SK gewünscht sei. 

\subsection{Bestehende explizite und implizite Aufträge} Aufträge von der Institution, sowie KM und S gestellt. 
\subsection{Settingwahl fürs Erstgespräch} Vorgaben KJPD und eigene Entscheidung.

