% Lernprozess & Reflexion der Arbeit
% **********************************
\section{Lernprozess \& Reflexion der Arbeit - 7 Seiten}\label{Reflexion}
In diesem letzten offiziellen Kapitel meiner Falldarstellung werde ich auch meinen persönlichen Lern- und Entwicklungsprozesses eingehen, sowie diesen kritisch reflektieren. In einem ersten Punkt werde ich auf die Falldarstellung in dieser Arbeit eingehen. In einem weiteren Punkt auf die eigene Entwicklung während der Fallarbeit.

\subsection{Reflexion der Falldarstellung} \label{sec:reflexionfalldarstellung}
Dies ist die erste grössere Falldarstellung, die ich im Rahmen meiner Weiterbildung geschrieben habe. Der Umfang von 28 Seiten für einen Fall schien mir zu Beginn etwas viel zu sein. Mir war im Vorfeld nicht klar, was ich da alles schreiben soll, ohne dass ich mich wiederhole und viel Theorie reinnehmen muss. Aufgrund der Inputs unseres Supervisors Patrick Wirz wurde mir schnell klar, dass ich mir einen ersten Raster schreiben musste, denn sonst würde ich mich verzetteln. Diesen erstellte ich, angelehnt an die Vorgaben der Wegleitung Version 2020.01 und schickte diese an meinen Supervisor zur ersten Prüfung. Im Anschluss erstellte ich mir einen Zeitplan, denn wir hatten klare Vorgaben, wann dieser Fall fertig sein muss. Dies gab mir eine erste Sicherheit, da ich die Übersicht hatte, wann welcher Teil fertig sein soll. Leider konnte ich meine Vorgabe nicht wirklich einhalten. Ich unterschätzte die fehlende Zeit, die sich durch zwei kleine Kinder zu Hause ergab. Mein Arbeitspensum, die Weiterbildung und die Familie zusammen waren insgesamt zu viel, um am Abend oder in der Freizeit konzentriert an der Arbeit schreiben zu können. Da meine Frau und ich die Betreuung der Kinder teilen, verbringe ich neben der Arbeit einen grossen Teil meiner Zeit mit meinen Kindern. Dies wollte und konnte ich auch nicht gross ändern, ohne dass wir als Eltern zusätzlich an unsere sonst schon strapazierte Grenzen gekommen wären. Glücklicherweise ergab es sich, dass ein drei Tage Block Weiterbildung als Reserve geplant war, falls eine Vorlesung ausfallen würde und ich den für das Schreiben meiner Arbeit verwenden konnte. An diesen drei Tagen hiess es nun so viel wie möglich erledigen. Im Vorfeld habe ich versucht soviel wie möglich vorzubereiten, so dass ich keine Zeit mehr mit Layout und sonstigen Aufgaben aufwenden musste. Dies zahlte sich aus, denn ich konnte diese Tage durchschreiben, war dann jedoch auch dementsprechend platt. 

Während dem Schreiben musste ich immer wieder meine Notizen aus der Fallarbeit konsultieren, da ich nicht mehr alles präsent im Kopf hatte. Dabei stellte sich heraus, dass die mehr oder weniger detaillierten Verlaufseinträge eine grosse Hilfe waren. Froh war ich zudem, dass ich während der Ausbildung damit begonnen hatte Hypothesen im Verlauf zu notieren und zwecks besserer Auffindbarkeit zu markieren. Somit konnte ich im Teil Hypothesen dies heraussuchen und verwenden. Weiter war ich froh, dass ich mir die Zieldefinition nach GAS notierte, somit konnte ich auch diese Notizen direkt für diese Arbeit verwenden. 

Anfänglich die grösste Schwierigkeit hatte ich bei dieser Arbeit mit dem Schreiben in der Ich-Form. Aufgrund meines Studiums in Psychologie und meiner vorherigen Schreibsozialisation im Bereich der Informatik, trainierte ich mir einen gewissen wissenschaftlichen Stil an. Ich konnte mich dabei an einer klar definierte Herangehensweise orientieren. Zudem war mir der Aufbau solcher Arbeiten geläufig und verständlich, da sie sich logisch für mich anfühlten. In dieser Arbeit jedoch stellte ich fest, dass mir nicht immer ganz klar war, in welche Kapitel die einzelnen Inhalte gehören. Zudem störte ich mich an den auf den ersten Blick häufigen Überschneidungen. Eine Intervention löst auch immer eine Interaktion aus. Somit hatte ich das Bedürfnis diese beiden zusammen zu halten. Für mich war das Auseinandernehmen des Therapieprozesses anstrengend und herausfordern. Dadurch konnte ich jedoch neue Aspekte im Fall SK erarbeiten und eine neue Perspektive einnehmen. 

Was würde ich für die nächste Arbeit anders machen? Im Grunde würde ich erneut wie oben beschrieben vorgehen. Für das nächste Mal möchte ich jedoch im Vorfeld bereits Zeit in form von mehreren Tagen am Stück reservieren, um dann die Arbeit zu schreiben. Mir wurde bewusst, dass ich mit zwei kleinen Kindern nicht die Illusion aufrecht halten kann, am Abend noch an der Arbeit zu schreiben. Dies führte nämlich zu einem nicht ausgeruhten Daddy, der wegen Kleinigkeiten zu schimpfen beginnt und sich danach dafür schämt. Vielmehr möchte ich mich für das Schreiben der Arbeit reservierte und mit meiner Frau abgesprochene Zeit ausserhalb der Familie nehmen. Es stellte sich heraus, dass eine Auszeit für meine Frau mit den Kindern für ein paar Tage bei den Schwiegereltern hilfreich ist, da dies uns beide entlastet. Für die nächste Arbeit möchte ich dies im Vorfeld einplanen. Weiter nehme ich mir vor, dass ich mich zu Beginn des Semesters, wenn ich mich für einen Fall für die Arbeit entschieden habe, immer wieder Notizen zu den einzelnen Themen machen möchte. Anstelle am Schluss die ganze Arbeit auf Einmal zu erstellen, das ganze Jahr immer wieder einzelne Teile davon zusammentragen. Da ich das Raster nun habe und ich die einzelnen Kapitel verstehe, sollte mir dies für die zweite Arbeit leichter fallen. 

\subsection{Reflexion der eigenen Entwicklung} \label{sec:reflexionEntwicklung}
Meine eigene Entwicklung möchte ich anhand der vier Kompetenzarten Personale Kompetenzen, Soziale- und kommunikative Kompetenzen, Fach- und Methodenkompetenzen, sowie Handlungs- und Umsetzungskompetenzen reflektieren. Aufgrund der Umfangsbegrenzung dieser Arbeit muss ich mich jedoch in den einzelnen Kompetenzarten wo immer möglich begrenzen, da dies sonst den Rahmen sprengen würde.

Im Bereich der personalen Kompetenz möchte ich auf die mir wichtigsten und persönlichsten Kompetenzen eingehen. Dabei spielt meine Erfahrung eine grosse Rolle, da ich bereits in zwei sehr unterschiedlichen Arbeitsbereichen gearbeitet habe (Handwerklich als Feinmechaniker und stark kognitiv als Softwareingenieur). Im Bereich meines Arbeitsalltags würde ich mich als authentisch bezeichnen. Auch wenn ich versuche die Lerninhalte aus der Ausbildung einfliessen zu lassen, so versuche ich meinem Stil treu zu bleiben. Dies ermöglich mir aus eigener Erfahrung im Umgang mit den Klienten, insbesonders den jugendlichen Klienten, rasch in ein Vertrauensbündnis zu gelangen. Bei SK half mir dies, das Vertrauen über die Zeit zu gewinnen, obwohl der Auftrag der Mutter zu Beginn sehr dominant war. Durch meine Anpassungsfähigkeit ist es mir zudem möglich, auf die unterschiedlichen Bedürfnisse der Klienten einzugehen, was mir bei SK geholfen hat auf ihre eigenen Ziele einzugehen und den Auftrag der Mutter für SK im Hintergrund zu belassen. Weiter bringe ich eine ausgeprägte Lernbereitschaft mit, die zusammen mit meinem Hang zum logischen Denken ein ständiges weiterbewegen auf dem Weg zum Psychotherapeuten anstrebt. Ich mag es, wenn ich neue Dinge lernen kann und diese dann auch zur Anwendung kommen. In diesem Fall geschehen durch den Fragebogen EB-45. Zudem würde ich mich als jemandem mit einer hohen Eigenverantwortung und einem hohen Verantwortungsbewusstsein sehen, was mir in diesem Umfeld am \ac{kjpd} zugute kommt, da ich die Fallverantwortung von SK habe und als Therapeut wohl auch verantwortungsvoll handeln will. Zusammen mit dem zielorientierten Handeln, welches ich intuitiv mitbringe kann ich davon ausgehen, dass wir in der Therapie immer auf irgendetwas hinarbeiten. Dies jedoch birgt auch eine gewisse Gefahr, da ich diese Zielorientierung intuitiv mitbringe, wende ich selten Instrumente wie der EB-45 an und habe sogar eine leichte Abneigung dagegen. Da aber gerne dazu lerne, werde ich den EB-45 zukünftig weiter verwenden, was mit einer gewissen Flexibilität einhergeht, da ich das Bedürfnis nacht intuitiven Handeln mit einem eher geplanten Handeln kombinieren kann. In den Supervisionsstunden, sowie auch in der Selbsterfahrung, konnte ich mich weiter im Bereich der Eigenreflektion üben, also eigene Denk-, Wahrnehmungs- und Handlungsmuster in der Interaktion mit der jeweiligen Gruppe zu reflektieren. In der Gruppensupervision ist mir aufgefallen, dass ich im Fall SK den Druck von der Mutter aufgrund meines Verantwortungsbewusstseins auf mich genommen habe, obwohl ich nicht für die Mutter verantwortlich bin. Für den Prozess, der die Mutter einbindet, jedoch nicht die Mutter. 

Die Beziehungsgestaltung zu SK veränderte sich einerseits durch die Dynamik des Falls, als sich SK selbständig meldete, jedoch auch aufgrund meiner bewusst konstruktiv und bedürfnisorientierten Beziehungsgestaltung. Mir wurde bei SK rasch klar, dass SK viel Verständnis für ihre Situation benötigte, da die Mutter dies anscheinend nicht konnte. Ich versuche SK an ihrem jeweiligen Standpunkt abzuholen, dabei versuche ich auf ihre jeweiligen Bedürfnisse einzugehen, ohne den Grund für ihre Anmeldung aus den Augen zu lassen. Zu Hilfe kommt mir in diesem Feld, dass ich gegenüber den Lebensentwürfe von SK neutral bleiben möchte. Was mir weniger gelungen ist, ist die Neutralität gegenüber der Mutter. Aufgrund der Gepflogenheit am \ac{kjpd} das Kind, die Jugendlichen ins Zentrum zu rücken, übernahm ich automatisch Partei für SK. Dies wurde mir auch in der Gruppensupervision bewusst, worauf ich mich anschliessend wieder vermehrt der Mutter gegeneüber neutraler zeigen konnte. Ein weiteres Anliegen von mir ist SK in ihrer Autonomie zu unterstützen. Dies versuche ich, indem ich SK immer wieder in Form von Aufträgen in die Handlung bringen möchte. Dies klappt bei SK ziemlich gut, da sie gewillt ist die besprochenen Themen umzusetzen. Gewöhnlich versuche ich mich mit involvierten Fachpersonen auszutauschen, um mir ein Bild von den Aufgabenbereichen der Klienten zu verschaffen. Bei SK telefonierte ich mit der Lehrperson, um mir ein Bild der schulischen Situation zu verschaffen. Dies half mir anschliessend mit SK das weitere Vorgehen zu erarbeiten. Nicht nur über die Gruppensupervision holte ich mir Feedback zu diesem Fall. Auch über die am \ac{kjpd} üblichen Fallcafés und Fallvorstellungsrunden habe ich mich mit Fachpersonen ausgetauscht. Dabei kam heraus, dass ich mir durchaus Zeit geben darf, mit SK in den Prozess zu gelangen. Ich war zu jenem Zeitpunkt etwas ungeduldig und erhoffte mir rascher eine Veränderung. Rückwirkend hätte mir wohl das Durchführen des EB-45 geholfen, um mir über den Fortschritt klarer zu werden. 

Die Fach- und Methodenkompetenz veränderte sich im Verlauf mit SK. Wie bereits mehrfach erwähnt startet ich meine therapeutische Laufbahn am \ac{kjpd} bevor ich mich am \ac{ief} in systemischer Psychotherapie weiterbildete. Mit dem fortschreitenden Wissenserwerb wurde mir bewusst bei SK, dass die Anliegen von SK in einem System entstanden sind, zu denen die Mutter, der Vater, die Geschwister und die Grosseltern gehören. Je mehr ich diese miteinbezogen habe in die Therapie, zumindest in Sensu, wurde es mir möglich die jeweiligen Interaktionen zu verstehen. Das Thema von SK mit ihrer Mutter wird wohl aus diesem neuen Verständnis entstanden sein, da ich darauf hellhöriger wurde. Zudem wurde mir aber auch die Begrenztheit von psychotherapeutischen Handlen immer wieder aufgezeigt (durch Vorlesung sowie Austausch mit Fachpersonen, sei es in den Vorlesungen oder den Supervisionsstunden). Weiter konnte ich feststellen, dass ich immer mehr Methoden in die Therapie mit SK einbezogen habe. Teilweise habe ich Methoden bereits angewendet, die mir aus dem Studium noch in Erinnerung waren (Vorlesung bei H. Grünwald), die dann in der Weiterbildung wieder vorkamen (Skalierungsfragen). In diesem Zusammenhang für mich schwierig sind die therapeutischen Wechselwirkungen, da ich mir dessen Punkte wie diagnostische Wechselwirkung und Interaktion während der Therapie nicht immer bewusst bin. Die Interaktionen gestalteten sich eher spontan und nicht geplant. Zeitweise ist mir nicht klar, welche Fortschritte wir erreichten, da ich mich stark an den Bedürfnissen von SK in den jeweiligen Stunden orientierte. Beim Schreiben dieser Arbeit wurde mir klar, dass ich teilweise den Fokus auf die ursprünglichen Themen der Auftragsklärung verloren hatte. 

Bezogen auf die Handlungs- und Umsetzungskompetenz konnte ich Verlauf gut beobachten, wie sich meine Kompetenzen in diesem Bereich veränderten. Fand die erste Auftragklärung noch ohne Wissen aus der Weitebrildung statt, so konnte ich bei der zweiten Auftragsklärung mit SK aus dem vollen schöpfen und das bereits gelernte einfliessen lassen. Das Hypothesenbilden kam dazu, was im Verlauf schön zu sehen ist, da ich diese direkt darin notierte. Zudem kamen Methoden wie das GAS, das zirkuläre Fragenstellen dazu, mit denen ich mir ein immer klareres Bild der Situation von SK erstellen konnte. Was aktuell wieder etwas in den Hintergrund gerückt ist, ist der aktive Einbezug des Systems, oder zumindest Teile davon. So möchte ich aus aktueller Erkenntnis aus dieser Arbeit die Mutter wenn möglich wieder zu einem Gespräch einladen. 

Abschliessend werde ich kurz über die das Kapitel Reflexion reflektieren. Dabei ist mir aufgefallen, dass ich mir diese vier eben behandelte Bereiche immer wieder aktiv vor Augen führen muss. Damit erhoffe ich mir auf dem restlichen Weg Orientierung auf meinem Weg insbesondere was mein Wissen, wo meine Lücken und wo meine blinden Flecken sind. Durch diese Arbeit und die Reflexion konnte ich mich erneut davon überzeugen, dass ich mich für den richtigen Weg entschieden habe, denn die Faszination und die Freude ist ungetrübt. Auch wenn die aktuelle Zeit aufgrund COVID-19 mir einen herben Dämpfer verpasste, da dies den Austausch massiv beeinträchtigte, tut dies kein Abbruch an der Methode. Ich freue mich auf die weiteren zwei Jahre und die hoffentlich noch mehrer Jahre als Therapeut in diesem Umfeld. SK wird eine wichtiger Schatz in meiner Truhe aus sich anhäufenden Erfahrungen.
