% Lernprozess & Reflexion der Arbeit
% **********************************
\section{Lernprozess \& Reflexion der Arbeit - 7 Seiten}\label{Reflexion}
In diesem letzten offiziellen Kapitel meiner Falldarstellung werde ich auch meinen persönlichen Lern- und Entwicklungsprozesses eingehen, sowie diesen kritisch reflektieren. In einem ersten Punkt werde ich auf die Falldarstellung in dieser Arbeit eingehen. In einem weiteren Punkt auf die eigene Entwicklung während der Fallarbeit.

\subsection{Reflexion der Falldarstellung} \label{sec:reflexionfalldarstellung}
Dies ist die erste grössere Falldarstellung, die ich im Rahmen meiner Weiterbildung geschrieben habe. Der Umfang von 28 Seiten für einen Fall schien mir zu Beginn etwas viel zu sein. Mir war im Vorfeld nicht klar, was ich da alles schreiben soll, ohne dass ich mich wiederhole und viel Theorie reinnehmen muss. Aufgrund der Inputs unseres Supervisors Patrick Wirz wurde mir schnell klar, dass ich mir einen ersten Raster schreiben musste, denn sonst würde ich mich verzetteln. Diesen erstellte ich, angelehnt an die Vorgaben der Wegleitung Version 2020.01 und schickte diese an meinen Supervisor zur ersten Prüfung. Im Anschluss erstellte ich mir einen Zeitplan, denn wir hatten klare Vorgaben, wann dieser Fall fertig sein muss. Dies gab mir eine erste Sicherheit, da ich die Übersicht hatte, wann welcher Teil fertig sein soll. Leider konnte ich meine Vorgabe nicht wirklich einhalten. Ich unterschätzte die fehlende Zeit, die sich durch zwei kleine Kinder zu Hause ergab. Mein Arbeitspensum, die Weiterbildung und die Familie zusammen waren insgesamt zu viel, um am Abend oder in der Freizeit konzentriert an der Arbeit schreiben zu können. Da meine Frau und ich die Betreuung der Kinder teilen, verbringe ich neben der Arbeit einen grossen Teil meiner Zeit mit meinen Kindern. Dies wollte und konnte ich auch nicht gross ändern, ohne dass wir als Eltern zusätzlich an unsere sonst schon strapazierte Grenzen gekommen wären. Glücklicherweise ergab es sich, dass ein drei Tage Block Weiterbildung als Reserve geplant war, falls eine Vorlesung ausfallen würde und ich den für das Schreiben meiner Arbeit verwenden konnte. An diesen drei Tagen hiess es nun so viel wie möglich erledigen. Im Vorfeld habe ich versucht soviel wie möglich vorzubereiten, so dass ich keine Zeit mehr mit Layout und sonstigen Aufgaben aufwenden musste. Dies zahlte sich aus, denn ich konnte diese Tage durchschreiben, war dann jedoch auch dementsprechend platt. 

Während dem Schreiben musste ich immer wieder meine Notizen aus der Fallarbeit konsultieren, da ich nicht mehr alles präsent im Kopf hatte. Dabei stellte sich heraus, dass die mehr oder weniger detaillierten Verlaufseinträge eine grosse Hilfe waren. Froh war ich zudem, dass ich während der Ausbildung damit begonnen hatte Hypothesen im Verlauf zu notieren und zwecks besserer Auffindbarkeit zu markieren. Somit konnte ich im Teil Hypothesen dies heraussuchen und verwenden. Weiter war ich froh, dass ich mir die Zieldefinition nach GAS notierte, somit konnte ich auch diese Notizen direkt für diese Arbeit verwenden. 

Anfänglich die grösste Schwierigkeit hatte ich bei dieser Arbeit mit dem Schreiben in der Ich-Form. Aufgrund meines Studiums in Psychologie und meiner vorherigen Schreibsozialisation im Bereich der Informatik, trainierte ich mir einen gewissen wissenschaftlichen Stil an. Ich konnte mich dabei an einer klar definierte Herangehensweise orientieren. Zudem war mir der Aufbau solcher Arbeiten geläufig und verständlich, da sie sich logisch für mich anfühlten. In dieser Arbeit jedoch stellte ich fest, dass mir nicht immer ganz klar war, in welche Kapitel die einzelnen Inhalte gehören. Zudem störte ich mich an den auf den ersten Blick häufigen Überschneidungen. Eine Intervention löst auch immer eine Interaktion aus. Somit hatte ich das Bedürfnis diese beiden zusammen zu halten. Für mich war das Auseinandernehmen des Therapieprozesses anstrengend und herausfordern. Dadurch konnte ich jedoch neue Aspekte im Fall SK erarbeiten und eine neue Perspektive einnehmen. 

Was würde ich für die nächste Arbeit anders machen? Im Grunde würde ich erneut wie oben beschrieben vorgehen. Für das nächste Mal möchte ich jedoch im Vorfeld bereits Zeit in form von mehreren Tagen am Stück reservieren, um dann die Arbeit zu schreiben. Mir wurde bewusst, dass ich mit zwei kleinen Kindern nicht die Illusion aufrecht halten kann, am Abend noch an der Arbeit zu schreiben. Dies führte nämlich zu einem nicht ausgeruhten Daddy, der wegen Kleinigkeiten zu schimpfen beginnt und sich danach dafür schämt. Vielmehr möchte ich mich für das Schreiben der Arbeit reservierte und mit meiner Frau abgesprochene Zeit ausserhalb der Familie nehmen. Es stellte sich heraus, dass eine Auszeit für meine Frau mit den Kindern für ein paar Tage bei den Schwiegereltern hilfreich ist, da dies uns beide entlastet. Für die nächste Arbeit möchte ich dies im Vorfeld einplanen. Weiter nehme ich mir vor, dass ich mich zu Beginn des Semesters, wenn ich mich für einen Fall für die Arbeit entschieden habe, immer wieder Notizen zu den einzelnen Themen machen möchte. Anstelle am Schluss die ganze Arbeit auf Einmal zu erstellen, das ganze Jahr immer wieder einzelne Teile davon zusammentragen. Da ich das Raster nun habe und ich die einzelnen Kapitel verstehe, sollte mir dies für die zweite Arbeit leichter fallen. 

\subsection{Reflexion der eigenen Entwicklung} \label{sec:reflexionEntwicklung}
Bezogen auf die vier Kompetenzarten gemäss Lehr- und Lernkonzept: Personale Kompetenzen, Soziale- und kommunikative Kompetenzen, Fach- und Methodenkompetenzen, sowie Handlungs- und Umsetzungskompetenzen.


Für mich am schwierigsten war das Kapitel \titleref{sec:TherapeutischeWechselwirkung}, da ich mir dessen Punkte wie diagnostische Wechselwirkung und Interaktion während der Therapie nicht immer bewusst war. Die Interaktionen gestalteten sich eher spontan und nicht geplant. Zeitweise war mir nicht klar, welche Fortschritte wir erreichten, da ich mich stark an den Bedürfnissen von SK in den jeweiligen Stunden orientierte. Beim Schreiben dieser Arbeit wurde mir klar, dass ich teilweise den Fokus auf die ursprünglichen Themen der Auftragsklärung verloren hatte. 
