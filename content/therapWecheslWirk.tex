% Therapeutische Wechselwirkung
%******************************
\section{Therapeutische Wechselwirkung - 4 Seiten} \label{sec:TherapeutischeWechselwirkung}
In diesem Kapitel werden Reaktionen auf Interventionen und Wendepunkte, sowie Veränderungen in den unterschiedlichen Systemen beschrieben und reflektiert. Zudem werde ich auf die Veränderung der Hypothesen und der Diagnostik eingehen. Weiter soll hiermit auch auf die wechselseitigen Anpassungsprozesse Rechnung getragen werden. 

\subsection{Diagnostische Wechselwirkung} 
Welche diagnostischen Annahmen habe ich im Verlauf getroffen und wie haben diese den therapeutischen Prozess beeinflusst? Eine Annahme habe ich aufgrund der Vordiagnose soziale Phobie getroffen und anhand dem Wunsch nach Stärkung des Selbstbewusstseins in Situationen mit mehreren Personen (Schule). Ich ging davon aus, dass Sk eine eher introvertierte Person sei und es deshalb SK schwer falle sich gegen aussen zu zeigen. Zudem schätzte ich sie als intelligente junge Frau ein, die sich sehr viele Gedanken über unterschiedliche Dinge im Leben machte, jedoch Mühe hatte ihre eigenen Stärken und Qualitäten zu sehen. SK erschien mir als eine Person, die sehr sensibel im zwischenmenschlichen Bereich ist, dabei viel wahrnimmt und ebenso viel Empathie für das Gegenüber aufbringen konnte, sich dabei jedoch oft zu wenig auf sich selber schaute. Ihre damals angewandte Strategie war es wann immer möglich, unangenehmen Situationen aus dem Weg zu gehen. Hilfreich in dieser Strategie war der Umstand, dass sie oft krank war und über starke Schmerzen klagte, weswegen sie vom Hausarzt eine Dispens bekam. Damals war noch nicht klar, dass sie an Endometriose litt. Meine Annahmen prägten mein therapeutisches Handeln insofern, als dass ich versuchte über die Beziehung, also das Vertrauen von SK mir gegenüber und einem sehr sachten pacing mittels emphatischen und wohlwollenden Vorgehen aufzubauen. Ich wollte ihr Selbstwert über das Stärken ihrer Ressourcen, indem sie darauf lösungsorientiert hilfreiche Handlungsmöglichkeiten erarbeiten konnte, aufbauen. Ich habe intuitiv gespürt, dass ich bei SK aufpassen musste, die Erwartungen der Mutter nicht zu übernehmen. Ich wollte SK darin stärken, dass sie selber die Themen bringen konnte, die sie bearbeiten wollte und dass ich sie in ihren Kompetenzen stärkte. 

Im Verlauf, insbesondere im zweiten Behandlungsteil, rückte das konflikthafte Familiensystem und die Beziehung zur Mutter vermehrt ins Zentrum. Versuchte ich ersten Teil ausschliesslich über die direkte therapeutische Beziehung zu arbeiten, ergänzte ich meine Interventionen immer mehr in Richtung eines systemischen Vorgehens mit dem Ziel, die Strukturen der Familie und des Umfelds von SK genauer zu verstehen. Dabei wendete ich vermehrt Techniken wie zirkuläre Fragen, GAS, Genogramm und das Bilden von Hypothesen an. Dies sollte SK ermöglichen ihre Familie aus einem anderen Blickwinkel betrachten zu können, um durch das Einnehmen einer Metaebene auf neue Lösungsideen zu kommen. 

Durch die starke Fokussierung auf das System verlor ich etwas den Fokus auf SK und ihre Schwierigkeiten im Umgang mit sozialen Situationen. Zudem ging ich von einem grossen Fortschritt diesbezüglich aus, da sie in der Vergangenheit viele früher schwierige Situationen gut meisterte. Als ein neuer Wiedereinstieg in die Schule anstand, reagierte SK mit vermehrt Schmerzen und depressiven Symptomen. Obwohl wir versuchten anhand der früher erarbeiteten Techniken eine Vorbereitung zu treffen, reichten diese Bemühungen nicht aus und SK scheiterte beim Versuch zurück in die Klasse zu kehren. Leider fanden aus unterschiedlichen Gründen bisher keine weiteren Termine mit SK statt. Ich vermute, dass ich SK mehr zugetraut habe, als sie in dieser Situation bewerkstelligen konnte. Dabei konnte ich nicht sehen, dass das Thema Familie bei SK in den Hintegrund rückte und der Wiedereinstieg sie in alte Muster abgleiten liess.


\subsection{Interaktion} Wechselwirkungen zwischen den beteiligten Personen und Systemen, insbesondere Fachpersonen- und Klient*innen-System beschreiben und reflektieren. Dazu gehört der Kontakt zur Mutter, den Lehrpersonen und den involvierten Fachpersonen wie Kinder \& Jugendsozialdienst, Hausarzt, Spezialarzt, sowie Interaktion zwischen Spital und Krankenkasse aufgrund mangelnder Versicherungsdeckung.



\subsection{Therapieergebnisse \& Wendepunkte} Was hat gewirkt. Was waren die Konsequenzen daraus? Eingehen auf Wendepunkte im Prozesse und Weg-Gabelungen, sowie Anzahl Sitzungen.




\subsection{Feedback} Rückmeldungen durch Klient*innensystem von Seiten der Mutter und SK. Hiermit soll auch eine Überleitung zum Kapitel \titleref{sec:Evaluationsverfahren} erfolgen.