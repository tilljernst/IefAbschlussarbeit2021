% Therapeutische Wechselwirkung
%******************************
\section{Therapeutische Wechselwirkung} \label{sec:TherapeutischeWechselwirkung}
In diesem Kapitel werden Reaktionen auf Interventionen und Wendepunkte, sowie Veränderungen in den unterschiedlichen Systemen beschrieben und reflektiert. Zudem werde ich auf die Veränderung der Hypothesen und der Diagnostik eingehen. Weiter soll hiermit auch auf die wechselseitigen Anpassungsprozesse Rechnung getragen werden. 

\subsection{Diagnostische Wechselwirkung} \label{sc:diagnostischeWechselwirkung}
Welche diagnostischen Annahmen habe ich im Verlauf getroffen und wie haben diese den therapeutischen Prozess beeinflusst? Eine Annahme habe ich aufgrund der Vordiagnose soziale Phobie getroffen und anhand dem Wunsch nach Stärkung des Selbstbewusstseins in Situationen mit mehreren Personen (Schule). Ich ging davon aus, dass Sk eine eher introvertierte Person sei und es deshalb SK schwer falle sich gegen aussen zu zeigen. Zudem schätzte ich sie als intelligente junge Frau ein, die sich sehr viele Gedanken über unterschiedliche Dinge im Leben machte, jedoch Mühe hatte ihre eigenen Stärken und Qualitäten zu sehen. SK erschien mir als eine Person, die sehr sensibel im zwischenmenschlichen Bereich ist, dabei viel wahrnimmt und ebenso viel Empathie für das Gegenüber aufbringen konnte, sich dabei jedoch oft zu wenig auf sich selber schaute. Ihre damals angewandte Strategie war es wann immer möglich, unangenehmen Situationen aus dem Weg zu gehen. Hilfreich in dieser Strategie war der Umstand, dass sie oft krank war und über starke Schmerzen klagte, weswegen sie vom Hausarzt eine Dispens bekam. Damals war noch nicht klar, dass sie an Endometriose litt. Meine Annahmen prägten mein therapeutisches Handeln insofern, als dass ich versuchte über die Beziehung, also das Vertrauen von SK mir gegenüber und einem sehr sachten Pacing mittels emphatischen und wohlwollenden Vorgehen aufzubauen. Ich wollte ihr Selbstwert über das Stärken ihrer Ressourcen, indem sie darauf lösungsorientiert hilfreiche Handlungsmöglichkeiten erarbeiten konnte, aufbauen. Ich habe intuitiv gespürt, dass ich bei SK aufpassen musste, die Erwartungen der Mutter nicht zu übernehmen. Ich wollte SK darin stärken, dass sie selber die Themen bringen konnte, die sie bearbeiten wollte und dass ich sie in ihren Kompetenzen stärkte. 

Im Verlauf, insbesondere im zweiten Behandlungsteil, rückte das konflikthafte Familiensystem und die Beziehung zur Mutter vermehrt ins Zentrum. Versuchte ich ersten Teil ausschliesslich über die direkte therapeutische Beziehung zu arbeiten, ergänzte ich meine Interventionen immer mehr in Richtung eines systemischen Vorgehens mit dem Ziel, die Strukturen der Familie und des Umfelds von SK genauer zu verstehen. Dabei wendete ich vermehrt Techniken wie zirkuläre Fragen, GAS, Genogramm und das Bilden von Hypothesen an. Dies sollte SK ermöglichen ihre Familie aus einem anderen Blickwinkel betrachten zu können, um durch das Einnehmen einer Metaebene auf neue Lösungsideen zu kommen. Daraus resultierte, dass sich SK begann gegenüber der Mutter zu behaupten. Weiter suchte sie aktiv nach alternativen Wohnmöglichkeiten, sei es bei der Verwandtschaft oder über den Kinder- \& Jugendsozialdienst. Der therapeutische Prozess verlagerte sich teilweise in die Richtung von sozialpsychiatrischem Arbeiten in Form von Gesprächen mit externen Fachpersonen. 


\subsection{Interaktion} \label{sc:interaktion}
Wechselwirkungen zwischen den beteiligten Personen und Systemen, insbesondere Fachpersonen- und Klient*innen-System beschreiben und reflektieren. Dazu gehört der Kontakt zur Mutter, den Lehrpersonen und den involvierten Fachpersonen wie Kinder \& Jugendsozialdienst, Hausarzt, Spezialarzt, sowie Interaktion zwischen Spital und Krankenkasse aufgrund mangelnder Versicherungsdeckung.

In der ersten Behandlungsphase war die Interaktion mit der Mutter neben der eigentlichen Therapie mit SK von zentraler Bedeutung. Rein rechtlich gesprochen konnte die Mutter als sorgeberechtigter Elternteil über den weiteren Verlauf der Therapie entscheiden. Für mich stellte dies eine Gratwanderung dar. Auf der einen Seite wollte ich die Beziehung zu SK aufbauen, weshalb ich die Anliegen der Mutter möglichst draussen halten musste, auf der anderen Seite involvierte ich die Mutter stark in den Prozess ein, indem wir zu dritt regelmässig Gespräche führten. Ich musste also nicht nur das Vertrauen von SK erarbeiten, sondern auch das der Mutter. Aus den Schilderungen der Mutter ging hervor, dass sie die ambulante Therapie bei meiner Vorgängerin aufgrund Differenzen zwischen ihr und der Therapeutin beendete. Deshalb war es wichtig, dass die Mutter wenigstens teilweise im Boot war. In der zweiten Behandlungsepisode meldete sich SK selbständig. Dabei wurde sie im Verlauf volljährig, was es bezogen auf die Mutter etwas einfacher machte. Ab diesem Zeitpunkt musste ich die Mutter nicht mehr so stark involvieren, da SK selbständig über die Therapie entscheiden konnte. 

Ein weitere wichtiger Interaktionspartner war und ist der Hausarzt. Dieser meldete sich in Absprache mit der Mutter in der zweiten Behandlungsepisode mit einem Video selbständig bei uns. Auf diesem Video war SK in einer Episode zu sehen, in der sie deutliche Bewegungsstörungen zeigte. Für mich war dies überraschend, da ich von SK bisher nichts über diese Vorfälle gehört habe. Zudem meldete sich der Hausarzt auf ausdrücklichen Wunsch der Mutter. Der Hausarzt hatte wohl SK darüber eingeweiht, nur war mir nicht klar wie einverstanden sie damit war. Als ich das Thema mit SK besprochen habe teilte SK mit mit, dass sie selber diese Episoden nicht so ernst genommen hat (obwohl diese bis zu 10mal am Tag aufgetreten sind). Zudem seien sie ihr peinlich gewesen, weswegen sie mir darüber nichts berichtete. Auf der einen Seite war ich froh, meldete sich der Hausarzt. Auf der anderen Seite störte diese Intervention den Therapieablauf, da die Medikamente gestoppt wurden, Termin mit SK und der Psychiaterin stattfinden mussten, sowie SK zu einem \ac{eeg} unterziehen musste. Später im Verlauf führte dies dazu, dass die Medikamentenvergabe zurück an den Hausarzt übergeben wurde, da sich die Psychiaterin am \ac{kjpd} scheute, ohne Langzeit-\ac{eeg} Medikamente zu verschreiben, da mit dem Absetzen der Medikamente die Bewegungs-Episoden stark abgenommen haben. Durch das aktive Melden des Hausarztes wurde auch die Interaktion mit der Mutter erneut verstärkt, da sie die Initiantin für das Vorgehen des Hausarztes gewesen ist. Somit rückte auch die Mutter wieder etwas näher an den Therapieprozess mit SK. In den darauf folgenden Stunden wurde die Möglichkeit gemeinsame Gespräche mit der Mutter zu führen mit SK besprochen. Was wiederum dazu führte, dass sich SK wieder näher an der Mutter orientierte.

Neben diesen Personen soll hier noch kurz auf die Interaktion mit den restlichen Personen eingegangen werden wie zum Beispiel der Lehrperson aus dem \ac{bvj}-Schuljahr. SK machte sich gleich zu Beginn ihrer Absenz grosse Sorgen um die Schule. Dabei stellte sich heraus, dass sich nicht getraute bei der Lehrperson anzurufen und Informationen über mögliche weiteren Schritte einzuholen. Zusammen erarbeiteten wir aus Seiten SK ein Verständnis gegenüber der Schule und dem möglichen Denkmustern anderer Leute. Obwohl die eigentliche Interaktion mit dem Lehrer sich auf ein Minimum beschränkte, so teilte dies dazu bei, dass sich SK über ihre Rolle als Schülerin bewusster und in die Absichten anderer Personen besser verstehen konnte. Eine weitere Interaktionsperson darf hier nicht fehlen und zwar ist dies der Vertreter des Kinder- \& Jugenddienstes, mit welchem wir gemeinsame Gespräche führten, um mögliche Wohn-Alternativen für SK zu besprechen. Diese Gespräche lösten Veränderungen zu Hause im System aus, denn SK liess Hinweise diesbezüglich gegenüber ihrer Mutter verlauten. SK zeigte sich zunehmend, was der Mutter auffiel und zu einer veränderung der Dynamik zwischen SK und ihrer Mutter führte. Dabei konnte die Mutter SK direkt sagen, dass sie SK nicht ziehen lassen wollte und es ihr am liebsten wäre, wenn SK noch bis zum 25. Lebensjahr bei ihr wohnen würde.

Ebenso die Unschöne Episode mit der Krankenkasse löste im System so einiges aus. In der Folge des verordneten Therapiestops von ganz oben, musste ich mich zwangsläufig mit der Mutter von SK in Verbindung setzen. Jedoch unter einem anderen Vorzeichen, denn diesmal musste ich der Mutter klar machen, dass dieses Verhalten so nicht tolerierbar sei und es zu den elterlichen Pflichten gehören würde, bei ausbildungspflichtigen Kindern für deren Unterhalt aufzukommen. Bisher waren die Interaktionen mit der Mutter von Anforderungen an mich geprägt. Diesmal wollte ich mich für das Anliegen von SK einsetzen. In der Folge konnte für SK eine eigene Krankenkassenpolice erstellt werden, was einen weiteren Schritt in Richtung Selbständigkeit darstellt. Dadurch, dass ich in diesen für die Mutter peinlichen Vorfall involviert wurde, nahm ich eine andere Position ein. Bisher schilderte mir die Mutter ihre Ängste und Sorgen bezüglich der Entwicklung von SK. Ihre eigene Rolle konnte sie bisher gekonnt ausklammern. Durch diesen Zwischenfall mit der Krankenkasse wurde der Mutter aufgezeigt, dass sie eine wichtige Rolle in der Dynamik von SK einnimmt. 
 

\subsection{Therapieergebnisse \& Wendepunkte} \label{sc:therapieergebnisse}
In diesem Abschnitt soll behandelt werden was gewirkt hat und was die Konsequenzen daraus waren. Zudem gehe ich auf Wendepunkte im Prozesse und Weg-Gabelungen, sowie Anzahl Sitzungen ein. 

Gewirkt hat aus meiner Sicht das Erarbeiten von Handlungsstrategien über die Orientierung an den Ressourcen. Dadurch war es SK möglich das gelernte direkt umzusetzen und dadurch positive Erfahrungen zu machen. Dieses Vorgehen blieb durchgehend bestehen und wurde immer wieder bei Bedarf angewendet. Zudem konnten Erkenntnisse über die Familiendynamik SK dazu verhelfen, eine neue Sichtweiser zu erhalten. Dies hat ihr aus meiner Sicht geholfen ihre Situation besser zu verstehen und einzuordnen. Scheint sie früher sich oft wenig Hilfreiche Gedanken gemacht zu haben, konnte sie mit Hilfe der systemischen Mitteln aus diesem Gedankenkarussel aussteigen und zeitweise eine Metaebene einnehmen. Eine grosse Hilfe scheint für SK auch die Möglichkeit der Reflexion zu sein, welche mit Hilfe der Skalierungsfragen permanent validiert werden konnte. Hilfreich könnte dabei die nicht-invasive Neugierde gewesen zu sein. Zudem scheint die bedingungslose Akzeptanz eine neue Erfahrung ermöglicht zu haben, da dies zu Hause nicht möglich ist und sich SK permanent in einer Schuld sieht. Hier hatte sie den Raum, als Person ernst genommen und mit Respekt auf Augenhöhe begegnet zu werden. Konsequenzen daraus sind, dass diese wirkungsvollen Mechanismen weiter angewendet werden.  

In der zweiten Behandlungsepisode verlor ich durch die starke Fokussierung auf das Familiensystem  den Fokus auf SK und ihre Schwierigkeiten im Umgang mit sozialen Situationen. Ich vernachlässigte diesen Punkt aufgrund grosser Fortschritt in der Vergangenheit. Dies wirkte sich meines Erachtens beim versuchten Wiedereinstieg in die Schule aus, indem SK mit vermehrt Schmerzen und depressiven Symptomen reagierte. Obwohl wir versuchten anhand der früher erarbeiteten Techniken eine Vorbereitung zu treffen, reichten diese Bemühungen nicht aus und SK scheiterte beim Versuch zurück in die Klasse zu kehren. Leider fanden aus unterschiedlichen Gründen bisher keine weiteren Termine mit SK statt. Ich vermute, dass ich SK mehr zugetraut habe, als sie in dieser Situation bewerkstelligen konnte. Dabei konnte ich nicht sehen, dass das Thema Familie bei SK in den Hintergrund rückte und der Wiedereinstieg sie in alte Muster abgleiten liess.

Wendepunkte sind auch in der Grafik in \textit{Abbildung\ref{fig:Behanldungsepisode}: \titleref{fig:Behanldungsepisode}} zu erkennen. Einerseits das Initiieren einer Therapiepause, in der für SK viel passierte (Schulabbruch, Diagnose der Endometriose, erfolglose Suche nach Lehrstelle). Nach dieser Pause, als SK sich aus eigenen Stücken für die Fortsetzung der Therapie aussprach und sich so für sich und ihre Bedürfnisse einsetzte. Die durchgeführte Operation, da sie sich gegenüber dem leitenden Arzt durchsetzen konnte, da dieser keine Operation in ihrem Alter durchführen wollte und es ihr danach besser ging. Weiter war der Vorfall mit der Krankenkasse für SK wohl hilfreicher als erst angenommen, da sie dadurch einen weiteren Schritt in Richtung Loslösung von zu Hause gehen konnte. 

Der aktuellste Wendepunkt scheint sich nach dem versuchten Wiedereintritt in die Schule einzuleiten. In der Vergangenheit blieb SK den Therapiterminen fern, wenn etwas damit nicht stimmte. Seit dem Versuch in die Schule zurückzugehen, kam keine eizige Therapiestunde zustande. SK meldete sich nicht von sich aus. Erst als ich mich aktiv über mehrere Kanälen bei ihr meldete, konnte sie Kontakt zu mir aufnehmen. Über diese Verhalten lässt sich gut Hypothesen generieren. Meine Vermutung ist a) sie schämt sich für den verpatzten Versuch und macht sich selber dabei fertig, b) sie misst der Therapie zu wenig Hilfepotential zu und denkt ans Aufhören.

Bezogen auf die Sitzungen kam SK bisher 40 mal in die Einzelstunde. Davon fanden knapp 10 in der ersten Behandlungsphase statt. Insgesamt fanden 5 Standorgespräche mit der Mutter zusammen statt, wobei alle in der ersten Phase inklusive Pause abgehalten wurden. SK fehlte in der ersten Phase 5 mal unentschuldigt, in der zweiten Phase 2 mal. Obwohl die zweite Phase länger war, fehlte sie weniger, was dafür zu sprechen scheint, dass ihr Commitment um einiges Grösser ist.

\subsection{Feedback} 
Rückmeldungen erfolgte durch das Klient*innensystem von Seiten der Mutter und von SK selber. In der ersten Phase, als die Mutter stärker eingebunden war, berichtete die Mutter von festgestellten Veränderungen bei SK zu Hause. SK könne sich besser für sich einsetzen und würde sich selständig bei Personen melden, wenn sie etwas von ihnen wolle. Früher wäre ihr das unangenehm gewesen und sie hätte sich davor gedrückt. Weitere Rückmeldungen von Seiten der Mutter blieben jedoch aus. Ich rechnete auch nicht damit, dass die Mutter weiter gross Rückmeldung zum Prozess mit SK geben wird, da sie aufgrund meiner Hypothese SK in einer Abhängigkeitsbeziehung behalten möchte und jegweilige positive bemerkte Veränderung bei SK diesem Ziel entgegenwirken muss.

SK selber betonte insbesondere am Anfang der zweiten Phase, wie sehr ihr die Gespräche geholfen haben. Bisher blieb - mit Ausnahme der Rückmeldung auf das GAS und EB-45 - die Rückmeldungen aus. Ich kann mir denken, dass dies ein gewohntes Muster im Bereich der Therapie ist. Es wird wohl eher selten vorkommen, dass Klient*innen unaufgefordert von sich aus positive Rückmeldungen geben. Wenn dann wohl eher wenn etwas nicht stimmt und auch diese wohl eher indirekt über Therapieabbruch oder ähnlichen versteckten Hinweisen. Ich gehe jedoch davon aus, dass SK von der Therapie aufgrund der Rückmeldung im GAS sowie im EB-45 profitieren konnte. Auf diese beiden Verfahren werden ich im nachfolgenden Kapitel \titleref{sec:Evaluationsverfahren} weiter eingehen.