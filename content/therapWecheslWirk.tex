% Therapeutische Wechselwirkung
%******************************
\section{Therapeutische Wechselwirkung} \label{sec:TherapeutischeWechselwirkung}
In diesem Kapitel werden Reaktionen auf Interventionen und Wendepunkte, sowie Veränderungen in den unterschiedlichen Systemen beschrieben und reflektiert. Zudem werde ich auf die Veränderung der Hypothesen und der Diagnostik eingehen. Weiter soll hiermit auch auf die wechselseitigen Anpassungsprozesse Rechnung getragen werden. 

\subsection{Diagnostische Wechselwirkung} \label{sc:diagnostischeWechselwirkung}
Welche diagnostischen Annahmen habe ich im Verlauf getroffen und wie haben diese den therapeutischen Prozess beeinflusst? Eine Annahme habe ich aufgrund der Vordiagnose soziale Phobie getroffen und anhand dem Wunsch nach Stärkung des Selbstbewusstseins in Situationen mit mehreren Personen (Schule). Ich ging davon aus, dass SK eine eher introvertierte Person ist und es deshalb SK schwer falle, sich gegen aussen zu zeigen. Zudem schätzte ich sie als intelligente junge Frau ein, die sich sehr viele Gedanken über unterschiedliche Dinge im Leben macht, jedoch Mühe hatte ihre eigenen Stärken und Qualitäten zu sehen. SK erschien mir als eine Person, die sehr sensibel im zwischenmenschlichen Bereich ist, viel wahrnimmt und ebenso viel Empathie für das Gegenüber aufbringen kann, jedoch oft zu wenig auf sich selber schaut. Ihre damals angewandte Strategie war es, unangenehmen Situationen aus dem Weg zu gehen. Dazu kam der Umstand, dass sie oft krank war und über starke Schmerzen klagte, weswegen ihr vom Hausarzt eine Dispens ausgestellt wurde. Damals war noch nicht klar, dass sie an Endometriose litt. Meine Annahmen prägten mein therapeutisches Handeln insofern, als dass ich versuchte über die Beziehung und einem sehr achtsamen Pacing vorzugehen. Ich versuchte ihren Selbstwert über das Stärken ihrer Ressourcen aufzubauen.  Intuitiv spürte ich, dass ich vorsichtig gegenüber den Erwartungen der Mutter sein musste. Ich wollte, dass SK ihre eigenen Themen bringen konnte. 

Im Verlauf, insbesondere im zweiten Behandlungsteil, rückte das konflikthafte Familiensystem und die Beziehung zur Mutter vermehrt ins Zentrum. Versuchte ich ersten Teil ausschliesslich über die direkte therapeutische Beziehung zu arbeiten, ergänzte ich meine Interventionen immer mehr in Richtung eines systemischen Vorgehens mit dem Ziel, die Strukturen der Familie und des Umfelds von SK genauer zu verstehen. Dabei wendete ich vermehrt systemische Techniken an. Dies sollte SK ermöglichen, ihre Familie aus einem anderen Blickwinkel betrachten zu können, um durch das Einnehmen einer Metaebene auf neue Lösungsideen zu kommen. Im weiteren Verlauf begann SK sich gegenüber der Mutter zu behaupten. Zudem suchte sie aktiv nach alternativen Wohnmöglichkeiten, sei es bei der Verwandtschaft oder über den Kinder- \& Jugendsozialdienst. Dabei verlagerte sich der therapeutische Prozess kurzfristig in den Bereich von sozialpsychiatrischem Arbeiten. 


\subsection{Interaktion} \label{sc:interaktion}
Im folgenden Kapitel werde ich die Wechselwirkungen zwischen den beteiligten Personen und Systeme, insbesondere Fachpersonen- und Klient*innen-Systeme, beschreiben und reflektieren. Dazu gehört der Kontakt zur Mutter, den Lehrpersonen und den involvierten Fachpersonen wie Kinder \& Jugendsozialdienst, Hausarzt, Spezialarzt, sowie der Interaktion zwischen Spital und Krankenkasse aufgrund mangelnder Versicherungsdeckung.

In der ersten Behandlungsphase war die Interaktion mit der Mutter von zentraler Bedeutung. Rein rechtlich gesprochen konnte die Mutter als sorgeberechtigter Elternteil über den weiteren Verlauf der Therapie entscheiden. Für mich stellte dies eine Gratwanderung dar. Auf der einen Seite wollte ich die Beziehung zu SK aufbauen, weshalb ich die Anliegen der Mutter möglichst draussen halten musste. Auf der anderen Seite involvierte ich die Mutter stark in den Prozess mit ein, indem wir zu dritt Gespräche führten. Ich musste also nicht nur das Vertrauen von SK erarbeiten, sondern auch das der Mutter. Aus den Schilderungen der Mutter ging hervor, dass sie die ambulante Therapie bei meiner Vorgängerin aufgrund Differenzen zwischen ihr und der Therapeutin beendete. Deshalb war es wichtig, dass die Mutter wenigstens teilweise im Boot war. In der zweiten Behandlungsepisode meldete sich SK selbständig. Ab diesem Zeitpunkt musste ich die Mutter nicht mehr so stark involvieren, da SK selbständig über die Therapie entscheiden konnte. 

Ein weitere wichtiger Interaktionspartner war und ist der Hausarzt. Dieser meldete sich in Absprache mit der Mutter in der zweiten Behandlungsepisode mit einem Video selbständig bei mir. Auf diesem Video war SK in einer Situation zu sehen, in der sie deutliche Bewegungsstörungen zeigte. Für mich war dies überraschend, da ich von SK bisher nichts über diese Vorfälle gehört habe. Zudem meldete sich der Hausarzt auf ausdrücklichen Wunsch der Mutter. Der Hausarzt hatte wohl SK darüber eingeweiht, nur war mir nicht klar wie einverstanden sie damit war. Als ich das Thema mit SK besprochen habe teilte SK mir mit, dass sie selber diese Vorfälle nicht so ernst genommen habe, obwohl diese bis zu zehnmal am Tag aufgetreten sind. Diese seien ihr peinlich gewesen, weswegen sie mir nichts sagte. Auf der einen Seite war ich froh, meldete sich der Hausarzt bei mir. Auf der anderen Seite unterbrach diser Vorfall den aktuellen Therapieprozess. Die Medikamente wurden gestoppt, es fanden Termine mit SK und der Psychiaterin statt und SK musste sich einem \ac{eeg} unterziehen. Später im Verlauf führte dies dazu, dass die Medikamentenvergabe an den Hausarzt übergeben wurde, da die Psychiaterin am \ac{kjpd} ohne Langzeit-\ac{eeg} bezüglich einer erneuten Abgabe von Medikamente zurückhaltend war. Denn mit dem Absetzen der Medikamente sind die Bewegungs-Episoden stark zurückgegangen. Durch das aktive Melden des Hausarztes wurde auch die Interaktion mit der Mutter erneut verstärkt, da sie die Initiantin für das Vorgehen des Hausarztes gewesen ist. Somit rückte auch die Mutter wieder etwas näher an den Therapieprozess mit SK, was ich jedoch begrüsste. In den darauf folgenden Stunden wurden mögliche Gespräche mit der Mutter besprochen. Dieser gesamte Prozess mit den Bewegungsmustern, dem Absetzten der Medikamente und der neurologischen Untersuchung führte dazu, dass sich SK erneut näher an der Mutter orientierte und sich von ihr unterstützen liess.

Neben diesen zentralen Personen soll hier noch kurz auf die Interaktion mit den restlichen Personen eingegangen werden wie zum Beispiel der Lehrperson aus dem \ac{bvj}-Schuljahr. SK machte sich gleich zu Beginn ihrer Absenz grosse Sorgen, da sie dies an die früheren Erfahrungen in der Schule erinnerte. Sie getraute sich nicht telefonisch Informationen über mögliche weiteren Schritte einzuholen. Wir erarbeiteten dabei ein Verständnis für das Handeln der Schule und setzten uns mit möglichen Denkmustern anderer Leute auseinander. Die eigentliche Interaktion mit den zuständigen Personen aus der Schule beschränkte sich auf ein Minimum. SK wurde sich jedoch über ihre Rolle als Schülerin bewusster und konnte die Absichten anderer Personen besser verstehen, was für sie entlastend wirkte. 

Eine weitere Person darf hier nicht fehlen und zwar ist dies der Vertreter des Kinder- \& Jugenddienstes, mit welchem wir gemeinsame Gespräche führten, um mögliche Wohn-Alternativen für SK zu besprechen. Diese Gespräche lösten Veränderungen zu Hause im System aus, denn SK liess gegenüber ihrer Mutter Bemerkungen über einen möglichen Auszug verlauten, was bei der Mutter Verlustängste auslöste. Dadurch veränderte sich die Dynamik zwischen SK und ihrer Mutter.

Ebenso gehört die Unschöne Episode mit der Krankenkasse in dieses Kapitel. In der Folge des verordneten Therapiestops, musste ich mich zwangsläufig mit der Mutter von SK in Verbindung setzen. Diesmal unter einem anderen Vorzeichen, denn ich war derjenige, der Forderungen stellen musste. In der Folge konnte für SK eine eigene Krankenkassenpolice erstellt werden, was einen weiteren Schritt auf dem Weg in die Richtung Selbständigkeit führte. Dadurch, dass ich in diesen für die Mutter peinlichen Vorfall involviert wurde, nahm ich eine andere Position ein. Bisher schilderte mir die Mutter ihre Ängste und Sorgen bezüglich der Entwicklung von SK. Ihre eigene Rolle konnte sie bisher gekonnt ausklammern. Nun wurde die Mutter aktiv in die Dynamik miteingebunden. 
 

\subsection{Therapieergebnisse \& Wendepunkte} \label{sc:therapieergebnisse}
In diesem Abschnitt werden die Interaktionen behandelt, die gewirkt haben und was die Konsequenzen daraus waren. Zudem gehe ich auf Wendepunkte im Prozesse, Weg-Gabelungen, sowie Anzahl Sitzungen ein. 

Gewirkt hat aus meiner Sicht das Erarbeiten von Handlungsstrategien über die Orientierung an den Ressourcen. Dadurch war es SK möglich das gelernte direkt umzusetzen und dadurch positive Erfahrungen zu machen. Dieses Vorgehen blieb durchgehend bestehen und wurde immer wieder angewendet. Zudem konnten Erkenntnisse über die Familiendynamik SK dazu verhelfen, eine neue Sichtweiser zu erhalten. Dies hat ihr geholfen ihre Situation besser zu verstehen und einzuordnen. Hatte SK früher dysfunktionale Gedanken, konnte sie mit Hilfe der systemischen Mitteln aus diesem Gedankenkarussel aussteigen und eine Metaebene einnehmen. Eine grosse Hilfe scheint für SK auch die Möglichkeit der Reflexion zu sein, welche mit Hilfe der Skalierungsfragen permanent validiert werden konnte. Hilfreich könnte dabei die nicht-invasive Neugierde gewesen sein. Zudem scheint die bedingungslose Akzeptanz eine neue Erfahrung für SK gewesen zu sein, da dies zu Hause nicht möglich ist und sich SK permanent in einer Schuld sieht. Hier hatte sie den Raum, als Person ernst genommen und mit Respekt auf Augenhöhe begegnet zu werden. Konsequenzen daraus sind, dass diese wirkungsvollen Mechanismen weiter angewendet werden.  

In der zweiten Behandlungsepisode verlor ich durch die starke Fokussierung auf das Familiensystem  den Fokus auf SK und ihre Schwierigkeiten im Umgang mit sozialen Situationen. Ich vernachlässigte diesen Punkt aufgrund grosser Fortschritte, welche SK diesbezüglich in der Vergangenheit gemacht hatte. Dies wirkte sich beim versuchten Wiedereinstieg in die Schule aus, indem SK mit vermehrt Schmerzen und depressiven Symptomen reagierte. Obwohl wir versuchten anhand der früher erarbeiteten Techniken eine Vorbereitung zu treffen, reichten diese Bemühungen nicht aus und SK scheiterte bei diesem neuen Versuch. Leider fanden aus unterschiedlichen Gründen bisher keine weiteren Termine mit SK statt. Ich vermute, dass ich SK mehr zugetraut habe, als sie in dieser Situation bewerkstelligen konnte. Dabei konnte ich nicht sehen, dass das Thema Familie bei SK in den Hintergrund rückte und der Wiedereinstieg in die Schule sie in alte Muster abgleiten liess.

Wendepunkte sind auch in der Grafik in \textit{Abbildung\ref{fig:Behanldungsepisode}: \titleref{fig:Behanldungsepisode}} zu erkennen. Einerseits das Initiieren einer Therapiepause, in der für SK viel passierte (Schulabbruch, Diagnose der Endometriose, erfolglose Suche nach Lehrstelle). Nach dieser Pause setzte SK sich für ihre Bedürfnisse ein. Die durchgeführte Operation, da sie sich gegenüber dem leitenden Arzt durchsetzen konnte, da dieser keine Operation in ihrem Alter durchführen wollte und es ihr danach besser ging. Weiter war der Vorfall mit der Krankenkasse für SK wohl hilfreicher als erst angenommen, da sie dadurch einen weiteren Schritt in Richtung Loslösung von zu Hause gehen konnte. 

Ein aktueller Wendepunkt scheint der versuchte Wiedereintritt in die Schule dazustellen. In der Vergangenheit blieb SK den Therapiterminen fern, wenn etwas nicht stimmte. Seit dem versuchten Wiedereintritt liess SK zwei Stunden unabgemeldet ausfallen. Zudem meldete sich auch nicht auf meine Textnachrichten. Erst als ich mich persisiteren über mehrere Kanälen bei ihr meldete, konnte sie erneut Kontakt zu mir aufnehmen. Meine Vermutung ist a) sie schämt sich vor mir für den verpatzten Wiedereinstieg und macht sich selber dabei fertig, b) sie misst der Therapie zu wenig Hilfepotential zu und denkt ans Aufhören.

Bisher kam SK 40 mal in die Einzelstunde. Davon fanden knapp 10 in der ersten Behandlungsphase statt. Insgesamt fanden 5 Standortgespräche mit der Mutter zusammen statt, alle in der ersten Phase (inklusive Pause). SK fehlte in der ersten Phase 5 mal unentschuldigt, in der zweiten Phase 2 mal. Obwohl die zweite Phase bereits länger ist, fehlte sie insgesamt weniger.

\subsection{Feedback} 
Rückmeldungen erfolgte durch das Klient*innensystem von Seiten der Mutter und von SK selber. In der ersten Phase, als die Mutter stärker eingebunden war, berichtete die Mutter von festgestellten Veränderungen bei SK zu Hause. SK könne sich besser für sich einsetzen und würde sich selbstständig bei Personen melden. Früher wäre ihr das unangenehm gewesen und sie hätte sich davor gedrückt. Weitere Rückmeldungen von Seiten der Mutter blieben jedoch aus. Ich rechnete auch nicht damit, dass die Mutter weiter gross Rückmeldung zum Prozess mit SK geben wird, da sie aufgrund meiner Hypothese SK in einer Abhängigkeitsbeziehung behalten möchte und positive bemerkte Veränderung bei SK diesem Ziel entgegenwirken würden.

SK betonte beim Wiedereinstieg in die Therapie, wie sehr ihr die Gespräche geholfen haben. Seitdem blieben weitere Rückmeldunge von Seiten SK aus (mit Ausnahme der Rückmeldungen aus dem \ac{gas} und dem EB-459. Ich gehe davon aus, dass dies ein übliches Muster in der Therapie ist. Es wird wohl eher selten vorkommen, dass Klient*innen unaufgefordert von sich aus Rückmeldungen geben. Wenn dann wohl eher bei Unstimmigkeiten und indirekt über einen Therapieabbruch. Ich gehe jedoch davon aus, dass SK von der Therapie aufgrund der Rückmeldung im GAS sowie im EB-45 profitieren konnte. Auf diese beiden Verfahren werden ich im nachfolgenden Kapitel \titleref{sec:Evaluationsverfahren} weiter eingehen.