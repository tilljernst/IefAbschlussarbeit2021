% Therapeutische Wechselwirkung
%******************************
\section{Therapeutische Wechselwirkung - 4 Seiten} \label{sec:TherapeutischeWechselwirkung}
In diesem Kapitel werden Reaktionen auf Interventionen und Wendepunkte, sowie Veränderungen in den unterschiedlichen Systemen beschrieben und reflektiert. Zudem werde ich auf die Veränderung der Hypothesen und der Diagnostik eingehen. Weiter soll hiermit auch auf die wechselseitigen Anpassungsprozesse Rechnung getragen werden. 

\subsection{Diagnostische Wechselwirkung} 
Welche diagnostischen Annahmen habe ich getroffen und wie haben diese den therap. Prozess beeinflusst.

Anhand meiner diagnostischen Annahmen die Wechselwirkungen des therapeutischen Prozesses rückblickend reflektieren (Diagnostik-therapeutischer Prozess). 



\subsection{Interaktion} Wechselwirkungen zwischen den beteiligten Personen und Systemen, insbesondere Fachpersonen- und Klient*innen-System beschreiben und reflektieren. Dazu gehört der Kontakt zur Mutter, den Lehrpersonen und den involvierten Fachpersonen wie Kinder \& Jugendsozialdienst, Hausarzt, Spezialarzt, sowie Interaktion zwischen Spital und Krankenkasse aufgrund mangelnder Versicherungsdeckung.



\subsection{Therapieergebnisse \& Wendepunkte} Was hat gewirkt. Was waren die Konsequenzen daraus? Eingehen auf Wendepunkte im Prozesse und Weg-Gabelungen, sowie Anzahl Sitzungen.




\subsection{Feedback} Rückmeldungen durch Klient*innensystem von Seiten der Mutter und SK. Hiermit soll auch eine Überleitung zum Kapitel \titleref{sec:Evaluationsverfahren} erfolgen.