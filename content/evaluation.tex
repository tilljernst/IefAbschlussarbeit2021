% Evaluationsverfahren
% ********************
\section{Evaluationsverfahren}\label{sec:Evaluationsverfahren}

\subsection{GAS - Goal Attainment Scale}
In diesem Kapitel werde ich auf ein Ziel, welches wir im Verlauf der Therapie mittels \ac{gas} mit SK behandelten, in Form einer ersten Übersicht, der Beschreibung des Ziels und der Kriterien, sowie einer eigenen Reflexion eingehen. Dabei habe ich ein Ziel ausgewählt, wobei zwei weitere mittels \ac{gas} erstellt wurden.

\subsubsection{Anwendung GAS}
Zu Beginn der zweiten Behandlungsepisode von SK erstellten wir mittels \ac{gas} gemäss \citeA{Kiresuk1968} eine erste explizite Zieldefinition. Bis zu diesem Zeitpunkt sammelten wir verschiedene Themen wie die kürzlich diagnostizierte Erkrankung und die damit einhergehenden Schmerzen, die depressive Verstimmung, welche in letzter Zeit zugenommen hat und das Schlafverhalten, da sie weder gut einschlafen noch durchschlafen konnte. Ein Thema schälte sich im Verlauf in der therapeutischen Arbeit mit SK immer deutlicher heraus: der Umgang mit der Mutter und die damit einhergehende schwierige Beziehung zu ihr. Nach einer kurzen Einführung meinerseits ins Thema Zielerreichung anhand konkreter anschaulicher Kriterien, liess ich SK entscheiden, ob sie sich auf ein solches Vorgehen einlassen wollte. Nachdem SK ihr Einverständnis gegeben hatte, wählten wir ein Ziel aus, welches für SK als das wichtigste definiert wurde und formulierten Kriterien dazu, anhand derer die Erreichung des Ziels oder auch eine mögliche Verschlechterung abzulesen ist. Dabei achtete ich darauf, dass es sich um ein individuelles Ziel von SK handelte, welches sich im Idealfall auf den Alltag von SK bezog. Zusammen mit SK versuchten wir das Ziel möglichst verständlich zu formulieren. Bei den Kriterien achtete ich auf gut überprüfbare Kriterien, die ich mit der Hilfe von SK gemäss \citeA{Dahling2006} in verschiedene Stufen einteilte. 

\subsubsection{Beschreibung}
TBD Thema

\subsubsection{Reflexion}
\subsection{EB-45/ILK}

\textit{Reflexion des persönlichen Lern- und Entwicklungsprozesses}
 