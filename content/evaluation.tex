% Evaluationsverfahren
% ********************
\section{Evaluationsverfahren}\label{sec:Evaluationsverfahren}

\subsection{GAS - Goal Attainment Scale}
In diesem Kapitel werde ich auf eine \ac{gas}-Zielsetzung in Form einer ersten Übersicht, der Beschreibung des Ziels und der Kriterien, sowie einer eigenen Reflexion eingehen. Dabei werde ich auf eine \ac{gas}-Skala genauer einergehen. 

\subsubsection{Anwendung GAS}
Zu Beginn der zweiten Behandlungsepisode von SK erstellten wir in einem ersten Schritt mittels \ac{gas} gemäss \citeA{Kiresuk1968} eine erste explizite Zieldefinition. Bis zu diesem Zeitpunkt sammelten wir verschiedene Themen wie die kürzlich diagnostizierte Erkrankung und die damit einhergehenden Schmerzen, die depressive Verstimmung, welche in letzter Zeit zugenommen hat und das Schlafverhalten, da sie weder gut einschlafen noch durchschlafen konnte. Ein Thema schälte sich im Verlauf in der therapeutischen Arbeit mit SK immer deutlicher heraus: der Umgang mit der Mutter und die damit einhergehende schwierige Beziehung zu ihr. Nach einer kurzen Einführung meinerseits ins Thema Zielerreichung anhand konkreter anschaulicher Kriterien, liess ich SK entscheiden, ob sie sich auf ein solches Vorgehen einlassen wollte. Nachdem SK ihr Einverständnis gegeben hatte, wählten wir ein Ziel aus, welches für SK als das wichtigste definiert wurde und formulierten Kriterien dazu, anhand derer die Erreichung des Ziels oder auch eine mögliche Verschlechterung abzulesen ist. Dabei achtete ich darauf, dass es sich um ein individuelles Ziel von SK handelte, welches sich im Idealfall auf den Alltag von SK bezog. Zusammen mit SK versuchten wir das Ziel möglichst verständlich zu formulieren. Bei den Kriterien achtete ich auf gut überprüfbare Kriterien, die ich mit der Hilfe von SK gemäss \citeA{Dahling2006} in verschiedene Stufen einteilte. In der Mitte der zweiten Behandlungsepisode in einem zweiten Schritt überprüften wir den Stand auf dem Weg zur Erreichung der Ziele. Zudem prüften wir, ob das Ziel für SK weiterhin relevant ist und ob SK weiter an diesem Ziel arbeiten möchte.

\subsubsection{Beschreibung}
Anhand eines \ac{gas}-Arbeitsblattes (siehe Anhang), welches wir vom \ac{ief} zur Verfügung bekommen haben, erarbeiteten wir das ausformulierte Ziel und die zugehörigen Kriterien. Dabei wurde der aktuellen Zustand bestimmt, sowie Kriterien, die zu einer Verbesserung, resp. Verschlechterung führen:

Ziel SK: \textit{Ich möchte mich gegenüber meiner Mutter frei fühlen}.
\begin{itemize}
   \item[+] 4-Linie: 1) \textit{SK kann auf ihr Zimmer gehen wann immer sie möchte, ohne dass ihre Mutter ihr folgt oder eine unpassende Bemerkung in Form eine Schuldzuweisung abgibt}, 2) \textit{SK kann je nach eigenem Bedürfnis Zeit mit sich alleine im Zimmer verbringen, ohne dass ihr Mutter sie dabei stört}, 3) \textit{die Mutter macht SK nicht mehr für die eigene Einsamkeit verantwortlich} und 4) \textit{wenn die Mutter an der Zimmertüre von SK klopft und SK kein Kontakt möchte, so entfernt sich die Mutter wieder}.
   \item[+] 2-Linie: 1) \textit{Es gibt Tage, an denen SK direkt und ohne sich zuerst mit der KM abzugeben auf ihr Zimmer begeben kann} und 2) \textit{die Mutter macht ausserhalb der Familie mit einer Freund*in ab}.
  \item[0] Linie: 1) \textit{SK kann nachdem sie nach Hause gekommen ist nicht direkt in ihr Zimmer gehen, ohne sich zuvor eine längere Zeit mit der Mutter abgegeben (unterhalten) zu haben}, 2) \textit{SK kann sich nicht lange in ihr Zimmer zurückziehen, ohne dass ihre Mutter ohne zu klopfen in ihr Zimmer kommt (Privatsphäre)}, 3) \textit{SK ist für die Mutter wie eine beste Freundin und muss mit ihr Zeit verbringen. Die Mutter hat neben SK keine Freunde} und 4) \textit{SK wird von ihrer Mutter für die Einsamkeit der Mutter verantwortlich gemacht}.
 \item[-] 2-Linie: 1) \textit{SK kann nicht mehr mit Freunden abmachen, weil es die Mutter verhindert} und 2) \textit{die Mutter würde die Türe zum Zimmer von SK entfernen, damit sie immer sieht, was SK macht}.
\end{itemize}
Nach ca. fünf Monaten überprüften wir das Ziel erneut. Dabei kam heraus, dass verschiedene Kriterien bereits erfolgreich erreicht wurden. Darunter folgende: 1) \textit{SK könne soviel Zeit auf ihrem Zimmer bleiben wie gewünscht}, 2)\textit{SK könne direkt nach dem Nachhausekommen auf ihr Zimmer gehen} und 3) \textit{die KM habe in der Zwischenzeit mit einer Freundin abgemacht}. Gemäss oben stehende Einteilung befindet sich SK dabei in den Kategorien der +4-Linie. SK möchte an diesem Ziel dran bleiben und lässt die Kategorien vorerst so bestehen. 

\subsubsection{Reflexion}


\subsection{EB-45/ILK}

\textit{Reflexion des persönlichen Lern- und Entwicklungsprozesses}
 