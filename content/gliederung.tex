% Therapeutische Wechselwirkung
%******************************

\section{Therapeutische Wechselwirkung - 4 Seiten} \label{TherapeutischeWechselwirkung}
\subsection{Diagnostische Wechselwirkung} Anhand meiner diagnostischen Annahmen die Wechselwirkungen des therapeutischen Prozesses rückblickend reflektieren (Diagnostik-therapeutischer Prozess). 
\subsection{Interaktion} Wechselwirkungen zwischen den beteiligten Personen und Systemen, insbesondere Fachpersonen- und Klient*innen-System beschreiben und reflektieren. Dazu gehört der Kontakt zur Mutter, den Lehrpersonen und den involvierten Fachpersonen wie Hausarzt und Spezialarzt, sowie Interaktion zwischen Spital und Krankenkasse aufgrund mangelnder Versicherungsdeckung.
\subsection{Therapieergebnisse} Was hat gewirkt. Was waren die Konsequenzen daraus?
\subsection{Wendepunkte} Eingehen auf Wendepunkte im Prozesse und Weg-Gabelungen, sowie Anzahl Sitzungen.
\subsection{Feedback} Rückmeldungen durch Klientinnensystem (S sowie KM). Hiermit soll auch eine Überleitung zum Kapitel \titleref{Evaluationsverfahren} erfolgen.

% Evaluationsverfahren
% ********************
\section{Evaluationsverfahren - 3 Seiten}\label{Evaluationsverfahren}
Anwendung, Beschreibung und Reflexion.
\subsection{GAS}
\subsection{EB-45/ILK}

\textit{Reflexion des persönlichen Lern- und Entwicklungsprozesses}

% Lernprozess & Reflexion der Arbeit
% **********************************
\section{Lernprozess \& Reflexion der Arbeit - 7 Seiten}\label{Reflexion}
\subsection{Falldarstellung} Kritische Reflexion des Lernprozesses in Auseinandersetzung mit der Falldarstellung.
\subsection{Eigene Entwicklung} Bezogen auf die vier Kompetenzarten gemäss Lehr- und Lernkonzept: Personale Kompetenzen, Soziale- und kommunikative Kompetenzen, Fach- und Methodenkompetenzen, sowie Handlungs- und Umsetzungskompetenzen.
