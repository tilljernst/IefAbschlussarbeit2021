% Therapeutischer Prozess
% ***********************
\section{Therapeutischer Prozess - 7 Seiten}\label{Prozess}
\subsection{Auftragsklärung} Dies soll anhand des Erstgesprächs, der laufenden Erneuerung der Auftragsklärung und anhand der Wiederanmeldung durch S beschrieben und reflektiert werden. Hier soll auch kurz auf das GAS eingegangen werden, welches weiter unten im Kapitel \titleref{Evaluationsverfahren} 	genauer beschrieben wird. 
\subsection{Beziehungsgestaltung} Beschreiben von Beziehungsaufbau und Gestaltung. Hier sollen auch Resonanzphänomene, die mit der Beziehungsgestaltung einhergehen, beschrieben werden. 
\subsection{Diagnostik} In diesem Kapitel soll auf die Diagnostik im Sinne von   Ressourcen, Beziehungen und Störungen, im Sinne von ICD Diagnosen, sowie bisherige Bewältigungsstrategien eingegangen werden. 
\subsection{Hypothesenbildung} Prozess der Hypthesenbildung und die im Laufe entstandenen Hypothesen sollen hier beschrieben und erläutert werden. 
\subsection{Interventionen} In diesem Kapitel sollen die durchgeführten Interventionen beschrieben und reflektiert werden. Dieses Kapitel leitet in das Kapitel \titleref{Verlauf} über.
\begin{itemize}
 \item Zirkuläre Fragen
 \item Lifeline mit S mit Theorie
 \item Aufstellungsarbeit
 \item Atmung gemäss Traumaseminar
 \item etc.
\end{itemize}
\subsubsection{Verlauf}\label{Verlauf} Im Verlauf sollen Reaktionen auf Interventionen und Wendepunkte, sowie Veränderungen in den unterschiedlichen Systemen beschrieben und reflektiert werden. Zudem soll auf die Veränderung der Hypothesen und der Diagnostik eingegangen werden. Rechnung soll hiermit auch auf die wechselseitigen Anpassungsprozesse getragen werden.

% Therapeutische Wechselwirkung
%******************************
\section{Therapeutische Wechselwirkung - 4 Seiten}
\subsection{Diagnostische Wechselwirkung} Anhand meiner diagnostischen Annahmen die Wechselwirkungen des therapeutischen Prozesses rückblickend reflektieren (Diagnostik-therapeutischer Prozess). 
\subsection{Interaktion} Wechselwirkungen zwischen den beteiligten Personen und Systemen, insbesondere Fachpersonen- und Klient*innen-System beschreiben und reflektieren. Dazu gehört der Kontakt zur Mutter, den Lehrpersonen und den involvierten Fachpersonen wie Hausarzt und Spezialarzt, sowie Interaktion zwischen Spital und Krankenkasse aufgrund mangelnder Versicherungsdeckung.
\subsection{Therapieergebnisse} Was hat gewirkt. Was waren die Konsequenzen daraus?
\subsection{Wendepunkte} Eingehen auf Wendepunkte im Prozesse und Weg-Gabelungen, sowie Anzahl Sitzungen.
\subsection{Feedback} Rückmeldungen durch Klientinnensystem (S sowie KM). Hiermit soll auch eine Überleitung zum Kapitel \titleref{Evaluationsverfahren} erfolgen.


% Evaluationsverfahren
% ********************
\section{Evaluationsverfahren - 3 Seiten}\label{Evaluationsverfahren}
Anwendung, Beschreibung und Reflexion.
\subsection{GAS}
\subsection{EB-45/ILK}

\textit{Reflexion des persönlichen Lern- und Entwicklungsprozesses}

% Lernprozess & Reflexion der Arbeit
% **********************************
\section{Lernprozess \& Reflexion der Arbeit - 7 Seiten}\label{Reflexion}
\subsection{Falldarstellung} Kritische Reflexion des Lernprozesses in Auseinandersetzung mit der Falldarstellung.
\subsection{Eigene Entwicklung} Bezogen auf die vier Kompetenzarten gemäss Lehr- und Lernkonzept: Personale Kompetenzen, Soziale- und kommunikative Kompetenzen, Fach- und Methodenkompetenzen, sowie Handlungs- und Umsetzungskompetenzen.
