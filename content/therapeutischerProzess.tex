% Therapeutischer Prozess
% ***********************
\section{Therapeutischer Prozess - 7 Seiten}\label{Prozess}
In diesem Kapitel werde ich auf den eigentlichen therapeutischen Prozess eingehen. Darin enthalten sind die Auftragsklärung, die Beziehungsgestaltung, die Diagnostik nach ICD Richtlinien, die Hypothesenbildung und Interventionen, sowie der Verlauf.   

Da es sich beim Fall SK um einen für mich im ambulanten Setting am \ac{kjpd} länger andauernden Fall handelt, der noch nicht abgeschlossen ist, möchte ich in einem ersten Schritt die bisherigen Stationen im Verlauf anhand einer grafischen Übersicht in Form eines Zeitstrahls abbilden. Dadurch soll die Orientierung für den Leser verständlicher werden und ermöglicht mir weiter unten im Kapitel \titleref{Verlauf} auf einzelne Abschnitte, sowie Ereignisse genauer eingehen zu können. Diese Orientierung soll auch im nächsten Kapitel \titleref{TherapeutischeWechselwirkung} Verwendung finden. 

\subsection{Fallübersicht}
TBD: Zeitstrahl

\subsection{Auftragsklärung}\label{Auftragsklärung} 
Gemäss \citeA{Wampold2015} gehört die Übereinstimmung der Ziele von Therapeut*in und Klient*in zu den allgemeinen Wirkfaktoren. Beim ersten \acf{akg} mit SK und ihrer Mutter verfolgte ich diesen Ansatz intuitiv. Wie ich mir aus dem Psychologiestudium aus den Vorlesungen im Bereich der humanistischen Therapie gemerkt habe (mit der Ausbildung am IEF habe ich erst kurz nach dem Gespräch angefangen), ist es wichtig die Anliegen und Wünsche der Klienten zu verstehen. 

Für das erste \ac{akg} hielt ich mich an der am \ac{kjpd} üblichen Vorgehensweise, welche den Ablauf des \acp{akg} grob umreisst und die einzelnen Punkte im Folgenden kursiv abgedruckt sind. Dazu gehört die \textit{Festlegung des Gesprächsziels}, welches dazu führen soll zu entscheiden, ob der \ac{kjpd} die richtige Stelle für das Anliegen ist. Weiter gehört auch eine kurze \textit{Vorstellung der Institution}, sowie die Schilderung des \textit{Ablaufs} dazu, bei dem auf den zeitlichen Rahmen und das grobe Vorgehen eingegangen wird. Bei der Schilderung von \textit{Anmeldegrund \& Problembeschreibung} wurde bereits klar, dass die Mutter die Anmeldung aus eigenem Wunsch vorgenommen hatte. Dabei sei es in der Vergangenheit zu Konflikten zwischen der Schule und SK gekommen, da diese krankheitsbedingt viele Fehltage hatte und die Schule dies nicht mehr länger akzeptieren wollte. Für die Problembeschreibung holte die Mutter weit aus und schilderte aus ihrer Sicht die Entstehung der Problematik, die zur Anmeldung geführt hätte. SK war während der gesamten Ausführung der Mutter erstaunlich ruhig. Sie meldete sich nicht aktiv zu Wort und korrigierte die Mutter auch nicht. Sie war wohl dabei, ich hatte jedoch die Vermutung, dass sie die Verantwortung des Gesprächsinhaltes ihrer Mutter übergeben hatte. Beim Abschnitt \textit{Reaktionen auf das Problemverhalten} ergaben sich Hinweise auf Mobbingsituationen an der Schule und einer vor ein paar Jahren stattgefundenen Trennung, auf die SK mit Schulangst und Rückzugstendenzen reagierte. Leider konnte ich bei diesem Gespräch weder von SK noch von ihrer Mutter Hinweise auf \textit{Erklärungsansätze} in Erfahrung bringen. Ich ging implizit davon aus, dass SK und die Mutter das Verhalten aufgrund der Mobbingerfahrung und der schwierigen Schulsituation in Verbindung brachten. Hier hätte ich aus aktueller Sicht das Bedürfnis die Anwesenden nach ihren eigenen Erklärungsansätze zu fragen. Bei den \textit{Bisherigen Massnahmen, reps. Lösungsversuche} wurde berichtet, dass SK nach einem erfolglosen Schulwechsel in eine Privatschule (aufgrund der Mobbingerfahrung), für ein paar Monate stationär in einer Klinik gewesen sei. Da sei die Diagnose Depression und eine zugehörige Angsterkrankung diagnostiziert worden. Im Anschluss an den Klinikaufenthalt wurde eine ambulante Nachbetreuung von einem ortsansässigen Psychiater in Form einer Therapeutin und einer ambulanten Spitex installiert. Die Mutter berichtete weiter, dass das Verhältnis zwischen SK und dem Psychiater sich verschlechterte und sie das Therapiebündnis beendete mit dem expliziten \textit{Anliegen} an mich, die ambulante Betreuung von SK zu übernehmen. Implizit hatte ich während dem Gespräch die Hypothese, dass es sich um mindestens einen weiteren Auftrag handelte, der im Zusammenhang mit der Schule stand. Als ich die Mutter darauf angesprochen habe bestätigte sie mir dies. Die Mutter wollte der Schule aufzeigen, dass sie gut für SK sorgen könne und die Schwierigkeiten von SK echt seien, da sie ja dann am \ac{kjpd} angebunden sei. Dadurch schien die Mutter mich als Verbündeten im Kampf gegen die Schule rekrutieren zu wollen. Bei SK konnte kein explizites Anliegen in Erfahrung gebracht werden. SK konnte sich auf mein Nachfragen durchringen das Thema Arbeit am Selbstvertrauen zu nennen. Als ich SK direkt auf die mir implizit erscheinenden Themen aus dem Klinikaufenthalt angesprochen habe, lehnte SK eine Unterstützung diesbezüglich mit der Begründung, selber damit klar zu kommen und über ausreichenden Strategien zu verfügen, ab. Ich akzeptierte dies, gab jedoch zu bedenken, dass mir der Auftrag für eine Therapieübernahme noch nicht klar war, weshalb ich Einzeltermine mit SK empfohlen habe, um die Auftragsklärung abzuschliessen und das weitere Vorgehen im Anschluss zusammen mit der Mutter bei einem Gespräch erneut zu besprechen. Darauf konnte sich die Mutter und SK einlassen.

Wie einleitend bereits erwähnt wurde, fand ein weiteres Auftragsklärungsgespräch im Verlauf der Behandlung auf Eigeninitiative von SK statt (SK wurde kurz darauf volljährig, weshalb ich mich dazu entschieden habe das \ac{akg} mit SK alleine und ohne Mutter zu führen). Zwei Monate zuvor schilderten mir SK und ihre Mutter bei einem Standortgespräch, welches aufgrund von COVID-19 über Telefon geführt wurde, dass SK keine Lehrstelle gefunden und sich für das 10te Schuljahr entschieden hätte. Somit hätten sie kein explizites Anliegen an eine weitere psychotherapeutische Behandlung und würden den Fall gerne abschliessen. Da ich den Fall noch nicht abgeschlossen hatte, entschied ich zusammen mit meinem Vorgesetzten den laufenden Fall zu verwenden und weiterzuführen. Zum Zeitpunkt des zweiten \acp{akg} befand ich mich bereits am Ende des ersten Jahres meiner Weiterbildung am \ac{ief}, weshalb ich meinen Fokus verstärkt auf die Passung der Ziele legte. Bereits in diesem zweiten Gespräch schilderte mir SK von ihrer aktuellen depressiven Verstimmtheit und den Schlafschwierigkeiten. Zudem hätte sie eine somatische Diagnose erhalten, welche sie sehr belasten würde. Der Wunsch von SK war es, über die depressive Verstimmtheit und die Krankheit zu sprechen, da sie bei der ersten Phase der Therapie bei mir von den Gesprächen profitieren konnte. SK erhoffte sich implizit eine Wiederholung der emotionalen Erleichterung. Zudem wollte SK über ihre Schlafschwierigkeiten sprechen, um einen Umgang damit zu finden. In diesem ersten Gespräch nach unserer Pause verfolgte ich das Ziel den Grund der Wiederanmeldung zu prüfen. Dabei wollte ich die expliziten Ziele noch nicht im Detail erarbeitet haben. Mir war es wichtig, dass SK aus eigener Motivation eine Therapie anstrebte und nicht im Hintergrund von der Mutter geschickt wurde, so wie ich es in der ersten Phase das Gefühl hatte. Deshalb entschieden wir uns weitere Gespräche zu führen, um die expliziten Aufträge konkret und messbar zu definieren. Dies erfolgte in den kommenden Sitzungen anhand einer detaillierten Zieldefinition mittels \ac{gas}, auf welches ich im Kapitel \titleref{Evaluationsverfahren} näher eingehen werde.

Die beiden \acp{akg} unterschieden sich nicht nur hinsichtlich meinem aktuellen Erfahrungsstand bezogen auf meine Weiterbildung und Therapieerfahrung, sondern auch hinsichtlich der Motivation von SK. War mir die Motivation von SK für eine Therapie in der ersten Behandlungsphase, initiiert durch die Mutter, nicht ganz klar, weswegen es zu deiner längeren Bedenkphase hinsichtlich Weiterführung der Therapie kam, so wurde diese durch die vorgebrachten Anliegen von SK spürbar. Standen beim ersten Gespräch die Anliegen der Mutter im Zentrum, so kamen beim Gespräch mit SK ihre eigenen Bedürfnisse zum Vorschein und leiteten die zweite Behandlungsphase ein. 


\subsection{Beziehungsgestaltung} \label{lbBeziehungsgestaltung}  
In diesem Kapitel werde ich auf die \titleref{lbBeziehungsgestaltung} in Form des Drei-Kreis-Modells der therapeutischen Beziehung gemäss \citeA{Borst2018} eingehen. Im äussersten Kreis des organisatorischen Rahmens handelt es sich beim \ac{kjpd} um eine öffentliche ambulante psychiatrische Institution, die ärztlich geführt wird. Das heisst aus meiner Sicht, dass grundsätzliche historisch bedingt eine störungsorientierte Sichtweise vorliegt, mit allem was dazu gehört und natürlich auch den klinischen Diagnosen nach ICD-10. Meine Rolle als Psychologe, dem ein anderes Menschenbild anhaftet, welches sich zu Beginn an einem humanistischen orientierte und mit Beginn meiner Ausbildung mit einem autonomieorientierten, sowie heteronomieorientierten Menschenbild ergänzte, wirkte dem tendenziell defizitären Menschenbild entgegen. Während meinem Studium veränderte sich auch mein therapeutische Grundhalten um ein in Richtung spezifisch-systemischer Grundhaltung. Dadurch, dass ich zu Beginn der Behandlung primär versuchte durch Neugier die Klientin und deren Anliegen zu verstehen, verfolgte ich eine nicht-invasive und freundliche therapeutische Grundtugend zu etablieren \cite{Cecchin1988}. Dies erfolgte intuitiv und ermöglichte mir ein erstes Vertrauen von SK zu gewinnen. Im Verlauf der Therapie und mit Beginn meiner Ausbildung am \ac{ief}ergänzte ich meine Grundhaltung, indem ich versuchte hypothesengeleitete Fragen zu stellen \cite{Andersen1990}, die für SK interessant und angemessen, jedoch ungewöhnlich waren, um SK auf neue Gedanken und ein vertieftes Verständnis über sich zu erlangen. Weiter folgte die Zirkularität, welche den Fokus der Problembeschreibung von SK als Interaktion in der Familie und im grösseren System einordnete und weg von einer individuellen Störung führte. Etwas distanziert betrachtet würde ich behaupten, dass mir diese sich verändernde Grundhaltung ermöglichte mit SK an den Punkt vorzustossen, an dem wir nun sind. Anfänglich konnte SK kein wirklich eigenes Anliegen benennen und es hatte den Anschein als wollte sie ihre Mutter ruhig stellen, konnte sie mit der Zeit genau dies Interaktion innerhalb der Familie und der Mutter als ein sie störendes Element identifizieren, was wiederum zu einer Vertiefung und Festigung der Beziehung zwischen SK und mir führte.



Bezogen auf den interaktionellen Rahmen ist in diesem Fall SK zu bemerken, dass sich auch dieser fortlaufend veränderte und anpasste. Mutete das Anliegen der Mutter an mich zu Beginn tendenziell nach Kontrolle über SK an, veränderte sich dies zunehmend in Richtung Hilfe zur Selbsthilfe für SK. Das Anliegen der Mutter, einerseits die Schule ruhig zu stellen, indem sie SK beim \ac{kjpd} anmeldete, und andererseits implizit SK hilflos zu halten, indem das Problem bei SK verortet wurde, trat diese Anliegen zunehmen in den Hintergrund und liess Platz für das Anliegen von SK. Dies erfolgte mit einer Abnahme des starken Einbezugs der Mutter mittels Standortgesprächen, die mit zunehmendem Erreichen der Volljährigkeit von SK stattgefunden hat. Dies ermöglichte zudem eine gewisse Übernahme von Verantwortung auf der Seite von SK, da sie nun selber für sich entscheiden musste, welche Themen sie bei mir behandeln wollte. Zudem meldete sich der Hausarzt von SK mit einem direkten Anliegen an mich, welches ich zuerst mit SK klären und besprechen musste. Diese von mir angesteuerte Transparenz führte wohl zudem dazu, dass mit SK immer wieder das Vertrauen aussprach und weiter regelmässig in die Therapie kommt. 

Innerhalb des innersten Bestimmungsstück der Beziehung zwischen SK und mir im der affektlogischen Rahmung verfolgte ich die Metastabilisierung eines instabilen Systems, ohne jedoch die Grundstruktur des Systems zu verändern. Bisher ist mir dies gelungen, jedoch gab es Episoden, an denen die Grundstruktur bei SK gefährdet wurde, indem sie sich für das Verlassen des familiären Systems ausgesprochen hatte, bei dem aus meiner Sicht eine zu starke Systemveränderung ins Rollen gebracht worden wäre, welche im Extremen mit dem Bruch der Familie, insbesondere der Mutter, einhergehen hätte können. Diesbezüglich fand jedoch eine erneute Beruhigung statt, auf der weiter aufgebaut werden kann. 

Es war mir nicht möglich diese drei Kreise gesondert zu betrachten, da sich diese  gegenseitig beeinflussen. Dadurch, dass ich SK bereits über mehrere Jahre begleite, veränderte sich einerseits bei Sarina die Aufträge dritter im organisatorischen Rahmen (obligatorische Schule, 10tes Schuljahr, Mutter), sowie das Anliegen von Dritten im interaktionellen Rahmen (Mutter, Hausarzt). Weiter wurde die Beziehung bezogen auf den organisatorischen Rahmen gestört, als die Krankenkasse SK infolge von ausstehenden Prämien auf die Schwarze Liste setzte und ich direkt von der Spitaladministration die Weisung eines sofortiger Behandlungsstop erhielt. In Absprache mit meinen Vorgesetzten konnte ich jedoch die Einzelstunden mit SK zu lasten des \acp{kjpd}weiterführen. Zudem hängt die Art der Beziehung auch von meiner Person ab, da ich meine eigene Geschichte, meine Erfahrung, mein persönlicher Kontext und mein Habitus zu einem gewissen Stück auch mit in die Beziehung hineinnehme. Durch die Vorgabe am \ac{kjpd} Supervisionsstunden zu nehmen, konnte ich gleich zu Beginn der Behandlung gewisse eigene Muster erkennen und reflektieren. Mit Beginn der Weiterbildung kamen weiter Selbsterfahrungsstunden hinzu und Gruppensupervisionsstunden, in welchen ich den Fall SK innerhalb der Gruppe besprochen habe und mir eine erweiterte Sichtweise ermöglichte, welche sich auf die Beziehung zwischen SK und mir auswirkte. All diese Veränderungen führten dazu, die Beziehung zwischen SK und mir zu gestalten und zu formen, welche aktuell von einem gegenseitigen Respekt und Vertrauen ersichtlich ist und SK ermöglicht mit ihren innersten Unsicherheiten in die Therapie zu kommen und dies auch zuzulassen. 

 
\subsection{Diagnostik} 
Gemäss dem psychiatrischen Austrittsbericht aus dem stationären Aufenthalt von SK geht hervor, dass SK die Kriterien einer sozialen Phobie gemäss ICD-10 F40.1 erfüllte. Dies führte insbesondere dazu, dass SK in exponierten Leistungssituationen soziale Ängste verspürte und sie mühe hatte auf wenig bekannte Menschen zuzugehen. Zudem wurde beim Klinikeintritt eine mittelgradige depressive Episode nach ICD-10 F32.1 diagnostiziert, bei welcher die Symptome sich im Verlauf der klinischen Behandlung verflüchtigten. 

Aufgrund meines Auftrags, resp. des zu Beginn fehlenden Auftrags, nahm ich die zuvor gestellten Diagnosen zur Kenntnis, ohne sie selber erneut zu überprüfen. Ich wollte SK den Raum geben eigene Themen zu finden und bei Bedarf zu besprechen. Den Fokus dabei legte ich auf ihre Ressourcen. SK zeigte von Beginn der Therapie in der ersten Behandlungsphase eine hohe Bereitschaft aktiv mitzumachen, was sich als eine grosse Ressource bei SK herausstellte. Ich spürte richtig, dass SK eine Veränderung anstreben wollte. Sie kam mit eigenen Themen und wählte das Thema Selbstbewusstsein, welches sie direkt anhand eines konkreten Beispiels in der Schule angehen wollte. Dabei wurde klar, dass SK bereits einige Strategien im Umgang mit Leistungssituationen erarbeitet hatte. Weiter viel mir auf, dass SK eine gute Reflexionsgabe mitbrachte. Dadurch konnten wir direkt ins Thema einsteigen, was unmittelbar in der Schule zu einem Erfolgserlebnis geführt hat. Im Verlauf der Behandlung wurden weitere Ressourcen von SK ersichtlich. Darunter die Neugierde, etwas neues auszuprobieren und ihre Offenheit für neue Erkenntnisse. Im Verlauf gelang es SK zunehmend schwierigere Themen zu identifizieren und zu benennen. Blieb sie zu Beginn tendenziell an der Oberfläche, so arbeitete sich kontinuierlich vor und brachte zunehmend schwierigere Themen hervor. Ihre Beharrlichkeit bei einem Thema dran zu bleiben gehört mitunter zu ihren Strategien. Aufgefallen ist mir, dass SK Themen, die wir in der Stunde besprochen haben, zu Hause weiter bearbeitete und erarbeitete Strategien versuchte umzusetzen.

 Im Verlauf zeigte sich bei SK wiederholend, dass sie auf sozial schwierige Situationen mit Rückzug reagiert. Schwierigkeiten bereiten ihr insbesondere Situationen mit vielen Leuten, da sie sich unsicher fühlt. SK hat einige wenige gute Sozialkontakte, mit denen sie ein freundschaftliches Verhältnis pflegt. Diese wenigen engen Kontakten konnten im Verlauf verstärkt werden, da diese SK gut tun und sie sich bei ihnen sicher fühlt. Dadurch kann sie Kraft tanken und sich von Stress erholen. Wichtig schein mir hier gut aufzupassen, dass diese wenigen sozialen Kontakte nicht abreissen, da sie für SK zudem halt geben. Zudem konnte ich eine Intensivierung der Beziehung zum mir als Therapeuten feststellen. SK schöpfte Kraft aus den Stunden und benötigte die Gespräche, um ihr Wohlbefinden zu stärken, was sie immer wieder betonte. Obwohl SK wenige sozialen Kontakte pflegt, schöpft sie daraus Kraft. In der Vergangenheit stand die Mutter oft stellvertretend für die fehlenden Freunde. 
 
 Bezogen auf die klinisch gestellte Diagnose einer Depression zeigten sich diese Symptome in letzter Zeit erneut. Dabei fällt auf, dass diese im längeren konflikthaften Umgang mit der Mutter und einer im Verlauf dazugekommenen somatischen Diagnose einer Endrometriose auftreten und in  Form von Rückzugstendenzen, Stimmungstiefs, Schlafschwierigkeiten und Interessenverlust verstärkt zeigen. Die somatischen Schmerzen, welche aufgrund der Erkrankung zyklisch auftreten, führen zudem dazu, dass sich SK vermehrt in ihr Zimmer zurückzieht. Dabei möchte sie niemandem zur Last fallen und meidet jeglichen sozialen Kontakt. In der Arbeit konnten wir feststellen, dass emotionaler Stress jeglicher Form, welcher jedoch hauptsächlich zu Hause im Umgang mit der Mutter auftritt, die Bauchschmerzen verstärken.
 
\subsection{Hypothesenbildung} 


Prozess der Hypothesenbildung und die im Laufe entstandenen Hypothesen sollen hier beschrieben und erläutert werden. 

\subsection{Interventionen} In diesem Kapitel sollen die durchgeführten Interventionen beschrieben und reflektiert werden. Dieses Kapitel leitet in das Kapitel \titleref{Verlauf} über.

\begin{itemize}
 \item Zirkuläre Fragen
 \item Lifeline mit S mit Theorie
 \item Aufstellungsarbeit
 \item Atmung gemäss Traumaseminar
 \item Skalierungsfragen
 \item etc.
\end{itemize}
\subsection{Verlauf}\label{Verlauf} Im Verlauf sollen Reaktionen auf Interventionen und Wendepunkte, sowie Veränderungen in den unterschiedlichen Systemen beschrieben und reflektiert werden. Zudem soll auf die Veränderung der Hypothesen und der Diagnostik eingegangen werden. Rechnung soll hiermit auch auf die wechselseitigen Anpassungsprozesse getragen werden.

