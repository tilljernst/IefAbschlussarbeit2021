% Therapeutischer Prozess
% ***********************
\section{Therapeutischer Prozess - 7 Seiten}\label{Prozess}
In diesem Kapitel werde ich auf den eigentlichen therapeutischen Prozess eingehen. Darin enthalten sind die Auftragsklärung, die Beziehungsgestaltung, die Diagnostik nach ICD Richtlinien, die Hypothesenbildung und Interventionen, sowie der Verlauf.   

Da es sich beim Fall SK um einen für mich im ambulanten Setting am \ac{kjpd} länger andauernden Fall handelt, der noch nicht abgeschlossen ist, möchte ich in einem ersten Schritt die bisherigen Stationen im Verlauf anhand einer grafischen Übersicht in Form eines Zeitstrahls abbilden. Dadurch soll die Orientierung für den Leser verständlicher werden und ermöglicht mir weiter unten im Kapitel \titleref{Verlauf} auf einzelne Abschnitte, sowie Ereignisse genauer eingehen zu können. Diese Orientierung soll auch im nächsten Kapitel \titleref{TherapeutischeWechselwirkung} Verwendung finden. 

\subsection{Fallübersicht}
TBD: Zeitstrahl

\subsection{Auftragsklärung}\label{Auftragsklärung} Dies soll anhand des Erstgesprächs, der laufenden Erneuerung der Auftragsklärung und anhand der Wiederanmeldung durch S beschrieben und reflektiert werden. Hier soll auch kurz auf das GAS eingegangen werden, welches weiter unten im Kapitel \titleref{Evaluationsverfahren} 	genauer beschrieben wird. 


\subsection{Beziehungsgestaltung} Beschreiben von Beziehungsaufbau und Gestaltung. Hier sollen auch Resonanzphänomene, die mit der Beziehungsgestaltung einhergehen, beschrieben werden. 
\subsection{Diagnostik} In diesem Kapitel soll auf die Diagnostik im Sinne von   Ressourcen, Beziehungen und Störungen, im Sinne von ICD Diagnosen, sowie bisherige Bewältigungsstrategien eingegangen werden. 
\subsection{Hypothesenbildung} Prozess der Hypthesenbildung und die im Laufe entstandenen Hypothesen sollen hier beschrieben und erläutert werden. 
\subsection{Interventionen} In diesem Kapitel sollen die durchgeführten Interventionen beschrieben und reflektiert werden. Dieses Kapitel leitet in das Kapitel \titleref{Verlauf} über.
\begin{itemize}
 \item Zirkuläre Fragen
 \item Lifeline mit S mit Theorie
 \item Aufstellungsarbeit
 \item Atmung gemäss Traumaseminar
 \item etc.
\end{itemize}
\subsection{Verlauf}\label{Verlauf} Im Verlauf sollen Reaktionen auf Interventionen und Wendepunkte, sowie Veränderungen in den unterschiedlichen Systemen beschrieben und reflektiert werden. Zudem soll auf die Veränderung der Hypothesen und der Diagnostik eingegangen werden. Rechnung soll hiermit auch auf die wechselseitigen Anpassungsprozesse getragen werden.

