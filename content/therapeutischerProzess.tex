% Therapeutischer Prozess
% ***********************
\section{Therapeutischer Prozess - 7 Seiten}\label{Prozess}
In diesem Kapitel werde ich auf den eigentlichen therapeutischen Prozess eingehen. Darin enthalten sind die Auftragsklärung, die Beziehungsgestaltung, die Diagnostik nach ICD Richtlinien, die Hypothesenbildung und Interventionen, sowie der Verlauf.   

Da es sich beim Fall SK um einen für mich im ambulanten Setting am \ac{kjpd} länger andauernden Fall handelt, der noch nicht abgeschlossen ist, möchte ich in einem ersten Schritt die bisherigen Stationen im Verlauf anhand einer grafischen Übersicht in Form eines Zeitstrahls abbilden. Dadurch soll die Orientierung für den Leser verständlicher werden und ermöglicht mir weiter unten im Kapitel \titleref{Verlauf} auf einzelne Abschnitte, sowie Ereignisse genauer eingehen zu können. Diese Orientierung soll auch im nächsten Kapitel \titleref{TherapeutischeWechselwirkung} Verwendung finden. 

\subsection{Fallübersicht}
TBD: Zeitstrahl

\subsection{Auftragsklärung}\label{Auftragsklärung} 
Gemäss \citeA{Wampold2015} gehört die Übereinstimmung der Ziele von Therapeut*in und Klient*in zu den allgemeinen Wirkfaktoren. Beim ersten \acf{akg} mit SK und ihrer Mutter verfolgte ich diesen Ansatz intuitiv. Wie ich mir aus dem Psychologiestudium aus den Vorlesungen im Bereich der humanistischen Therapie gemerkt habe (mit der Ausbildung am IEF habe ich erst kurz nach dem Gespräch angefangen), ist es wichtig die Anliegen und Wünsche der Klienten zu verstehen. 

Für das erste \ac{akg} hielt ich mich an der am \ac{kjpd} üblichen Vorgehensweise, welche den Ablauf des \acp{akg} grob umreisst und die einzelnen Punkte im Folgenden kursiv abgedruckt sind. Dazu gehört die \textit{Festlegung des Gesprächsziels}, welches dazu führen soll zu entscheiden, ob der \ac{kjpd} die richtige Stelle für das Anliegen ist. Weiter gehört auch eine kurze \textit{Vorstellung der Institution}, sowie die Schilderung des \textit{Ablaufs} dazu, bei dem auf den zeitlichen Rahmen und das grobe Vorgehen eingegangen wird. Bei der Schilderung von \textit{Anmeldegrund \& Problembeschreibung} wurde bereits klar, dass die Mutter die Anmeldung aus eigenem Wunsch vorgenommen hatte. Dabei sei es in der Vergangenheit zu Konflikten zwischen der Schule und SK gekommen, da diese krankheitsbedingt viele Fehltage hatte und die Schule dies nicht mehr länger akzeptieren wollte. Für die Problembeschreibung holte die Mutter weit aus und schilderte aus ihrer Sicht die Entstehung der Problematik, die zur Anmeldung geführt hätte. SK war während der gesamten Ausführung der Mutter erstaunlich ruhig. Sie meldete sich nicht aktiv zu Wort und korrigierte die Mutter auch nicht. Sie war wohl dabei, ich hatte jedoch die Vermutung, dass sie die Verantwortung des Gesprächsinhaltes ihrer Mutter übergeben hatte. Beim Abschnitt \textit{Reaktionen auf das Problemverhalten} ergaben sich Hinweise auf Mobbingsituationen an der Schule und einer vor ein paar Jahren stattgefundenen Trennung, auf die SK mit Schulangst und Rückzugstendenzen reagierte. Leider konnte ich bei diesem Gespräch weder von SK noch von ihrer Mutter Hinweise auf \textit{Erklärungsansätze} in Erfahrung bringen. Ich ging implizit davon aus, dass SK und die Mutter das Verhalten aufgrund der Mobbingerfahrung und der schwierigen Schulsituation in Verbindung brachten. Hier hätte ich aus aktueller Sicht das Bedürfnis die Anwesenden nach ihren eigenen Erklärungsansätze zu fragen. Bei den \textit{Bisherigen Massnahmen, reps. Lösungsversuche} wurde berichtet, dass SK nach einem erfolglosen Schulwechsel in eine Privatschule (aufgrund der Mobbingerfahrung), für ein paar Monate stationär in einer Klinik gewesen sei. Da sei die Diagnose Depression und eine zugehörige Angsterkrankung diagnostiziert worden. Im Anschluss an den Klinikaufenthalt wurde eine ambulante Nachbetreuung von einem ortsansässigen Psychiater in Form einer Therapeutin und einer ambulanten Spitex installiert. Die Mutter berichtete weiter, dass das Verhältnis zwischen SK und dem Psychiater sich verschlechterte und sie das Therapiebündnis beendete mit dem expliziten \textit{Anliegen} an mich, die ambulante Betreuung von SK zu übernehmen. Implizit hatte ich während dem Gespräch die Hypothese, dass es sich um mindestens einen weiteren Auftrag handelte, der im Zusammenhang mit der Schule stand. Als ich die Mutter darauf angesprochen habe bestätigte sie mir dies. Die Mutter wollte der Schule aufzeigen, dass sie gut für SK sorgen könne und die Schwierigkeiten von SK echt seien, da sie ja dann am \ac{kjpd} angebunden sei. Dadurch schien die Mutter mich als Verbündeten im Kampf gegen die Schule rekrutieren zu wollen. Bei SK konnte kein explizites Anliegen in Erfahrung gebracht werden. SK konnte sich auf mein Nachfragen durchringen das Thema Arbeit am Selbstvertrauen zu nennen. Als ich SK direkt auf die mir implizit erscheinenden Themen aus dem Klinikaufenthalt angesprochen habe, lehnte SK eine Unterstützung diesbezüglich mit der Begründung, selber damit klar zu kommen und über ausreichenden Strategien zu verfügen, ab. Ich akzeptierte dies, gab jedoch zu bedenken, dass mir der Auftrag für eine Therapieübernahme noch nicht klar war, weshalb ich Einzeltermine mit SK empfohlen habe, um die Auftragsklärung abzuschliessen und das weitere Vorgehen im Anschluss zusammen mit der Mutter bei einem Gespräch erneut zu besprechen. Darauf konnte sich die Mutter und SK einlassen.

Wie einleitend bereits erwähnt wurde, fand ein weiteres Auftragsklärungsgespräch im Verlauf der Behandlung auf Eigeninitiative von SK statt (SK wurde kurz darauf volljährig, weshalb ich mich dazu entschieden habe das \ac{akg} mit SK alleine und ohne Mutter zu führen). Zwei Monate zuvor schilderten mir SK und ihre Mutter bei einem Standortgespräch, welches aufgrund von COVID-19 über Telefon geführt wurde, dass SK keine Lehrstelle gefunden und sich für das 10te Schuljahr entschieden hätte. Somit hätten sie kein explizites Anliegen an eine weitere psychotherapeutische Behandlung und würden den Fall gerne abschliessen. Da ich den Fall noch nicht abgeschlossen hatte, entschied ich zusammen mit meinem Vorgesetzten den laufenden Fall zu verwenden und weiterzuführen. Zum Zeitpunkt des zweiten \acp{akg} befand ich mich bereits am Ende des ersten Jahres meiner Weiterbildung am \ac{ief}, weshalb ich meinen Fokus verstärkt auf die Passung der Ziele legte. Bereits in diesem zweiten Gespräch schilderte mir SK von ihrer aktuellen depressiven Verstimmtheit und den Schlafschwierigkeiten. Zudem hätte sie eine somatische Diagnose erhalten, welche sie sehr belasten würde. Der Wunsch von SK war es, über die depressive Verstimmtheit und die Krankheit zu sprechen, da sie bei der ersten Phase der Therapie bei mir von den Gesprächen profitieren konnte. SK erhoffte sich implizit eine Wiederholung der emotionalen Erleichterung. Zudem wollte SK über ihre Schlafschwierigkeiten sprechen, um einen Umgang damit zu finden. In diesem ersten Gespräch nach unserer Pause verfolgte ich das Ziel den Grund der Wiederanmeldung zu prüfen. Dabei wollte ich die expliziten Ziele noch nicht im Detail erarbeitet haben. Mir war es wichtig, dass SK aus eigener Motivation eine Therapie anstrebte und nicht im Hintergrund von der Mutter geschickt wurde, so wie ich es in der ersten Phase das Gefühl hatte. Deshalb entschieden wir uns weitere Gespräche zu führen, um die expliziten Aufträge konkret und messbar zu definieren. Dies erfolgte in den kommenden Sitzungen anhand einer detaillierten Zieldefinition mittels \ac{gas}, auf welches ich im Kapitel \titleref{Evaluationsverfahren} näher eingehen werde.

Die beiden \acp{akg} unterschieden sich nicht nur hinsichtlich meinem aktuellen Erfahrungsstand bezogen auf meine Weiterbildung und Therapieerfahrung, sondern auch hinsichtlich der Motivation von SK. War mir die Motivation von SK für eine Therapie in der ersten Behandlungsphase, initiiert durch die Mutter, nicht ganz klar, weswegen es zu deiner längeren Bedenkphase hinsichtlich weiterführung der Therapie kam, so wurde diese durch die vorgebrachten Anliegen von SK spürbar. Standen beim ersten Gespräch die Anliegen der Mutter im Zentrum, so kamen beim Gespräch mit SK ihre eigenen Bedürfnisse zum Vorschein und leiteten die zweite Behandlungsphase ein. 


\subsection{Beziehungsgestaltung} Beschreiben von Beziehungsaufbau und Gestaltung. Hier sollen auch Resonanzphänomene, die mit der Beziehungsgestaltung einhergehen, beschrieben werden.

 
\subsection{Diagnostik} In diesem Kapitel soll auf die Diagnostik im Sinne von   Ressourcen, Beziehungen und Störungen, im Sinne von ICD Diagnosen, sowie bisherige Bewältigungsstrategien eingegangen werden.
 
\subsection{Hypothesenbildung} Prozess der Hypthesenbildung und die im Laufe entstandenen Hypothesen sollen hier beschrieben und erläutert werden. 

\subsection{Interventionen} In diesem Kapitel sollen die durchgeführten Interventionen beschrieben und reflektiert werden. Dieses Kapitel leitet in das Kapitel \titleref{Verlauf} über.

\begin{itemize}
 \item Zirkuläre Fragen
 \item Lifeline mit S mit Theorie
 \item Aufstellungsarbeit
 \item Atmung gemäss Traumaseminar
 \item Skalierungsfragen
 \item etc.
\end{itemize}
\subsection{Verlauf}\label{Verlauf} Im Verlauf sollen Reaktionen auf Interventionen und Wendepunkte, sowie Veränderungen in den unterschiedlichen Systemen beschrieben und reflektiert werden. Zudem soll auf die Veränderung der Hypothesen und der Diagnostik eingegangen werden. Rechnung soll hiermit auch auf die wechselseitigen Anpassungsprozesse getragen werden.

